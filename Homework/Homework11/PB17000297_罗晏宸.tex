\documentclass[11pt]{article}
\usepackage[UTF8]{ctex}

%%%%%%%%%%%%%%%%%%%%%%%%%%%%%%%%%%%%%%%%%
% Cleese Assignment
% Structure Specification File
% Version 1.0 (27/5/2018)
%
% This template originates from:
% http://www.LaTeXTemplates.com
%
% Author:
% Vel (vel@LaTeXTemplates.com)
%
% License:
% CC BY-NC-SA 3.0 (http://creativecommons.org/licenses/by-nc-sa/3.0/)
%
%%%%%%%%%%%%%%%%%%%%%%%%%%%%%%%%%%%%%%%%%

%----------------------------------------------------------------------------------------
%	PACKAGES AND OTHER DOCUMENT CONFIGURATIONS
%----------------------------------------------------------------------------------------

\usepackage{lastpage} % Required to determine the last page number for the footer

\usepackage{graphicx} % Required to insert images

\setlength\parindent{0pt} % Removes all indentation from paragraphs

\usepackage[most]{tcolorbox} % Required for boxes that split across pages

\usepackage{booktabs} % Required for better horizontal rules in tables

\usepackage{listings} % Required for insertion of code

\usepackage{etoolbox} % Required for if statements

\usepackage{amsmath}
\usepackage{amsthm}
\usepackage{amssymb}
\usepackage{indentfirst}
\usepackage{diagbox}
\usepackage{subfigure}
\usepackage{float}
\usepackage{xcolor}
\usepackage[colorlinks, linkcolor = black]{hyperref}

\usepackage{enumerate}
\usepackage{enumitem}
\setlist{
    leftmargin = .1\linewidth,
    % rightmargin = .1\linewidth,
    % label=\emph{\alph*}.
}

\setlength{\parindent}{2em}

\usepackage{siunitx}
\sisetup
{
    output-exponent-marker = \ensuremath{\mathrm{E}},
    exponent-product = {},
    retain-explicit-plus,
    retain-zero-exponent,
}
%----------------------------------------------------------------------------------------
%	MARGINS
%----------------------------------------------------------------------------------------

\usepackage{geometry} % Required for adjusting page dimensions and margins

\geometry{
    paper=a4paper, % Change to letterpaper for US letter
    top=3cm, % Top margin
    bottom=3cm, % Bottom margin
    left=2.5cm, % Left margin
    right=2.5cm, % Right margin
    headheight=14pt, % Header height
    footskip=1.4cm, % Space from the bottom margin to the baseline of the footer
    headsep=1.2cm, % Space from the top margin to the baseline of the header
    %showframe, % Uncomment to show how the type block is set on the page
}

%----------------------------------------------------------------------------------------
%	FONT
%----------------------------------------------------------------------------------------

\usepackage[utf8]{inputenc} % Required for inputting international characters
\usepackage[T1]{fontenc} % Output font encoding for international characters

% \usepackage[sfdefault,light]{roboto} % Use the Roboto font

%----------------------------------------------------------------------------------------
%	HEADERS AND FOOTERS
%----------------------------------------------------------------------------------------

\usepackage{fancyhdr} % Required for customising headers and footers

\pagestyle{fancy} % Enable custom headers and footers

\lhead{\small\assignmentClass} % Left header; output the instructor in brackets if one was set
\chead{\small\assignmentTitle} % Centre header
\rhead{\small\ifdef{\assignmentAuthorName}{\assignmentAuthorName}{\ifdef{\assignmentDate}{Due\ \assignmentDate}{}}} % Right header; output the author name if one was set, otherwise the due date if that was set

\lfoot{} % Left footer
\cfoot{\small Page\ \thepage\ of\ \pageref{LastPage}} % Centre footer
\rfoot{} % Right footer

\renewcommand\headrulewidth{0.5pt} % Thickness of the header rule

%----------------------------------------------------------------------------------------
%	MODIFY SECTION STYLES
%----------------------------------------------------------------------------------------

\usepackage{titlesec} % Required for modifying sections

%------------------------------------------------
% Section

\titleformat
{\section} % Section type being modified
[block] % Shape type, can be: hang, block, display, runin, leftmargin, rightmargin, drop, wrap, frame
{\Large\bfseries} % Format of the whole section
{\arabic{section}} % Format of the section label
{6pt} % Space between the title and label
{} % Code before the label

\titlespacing{\section}{0pt}{0.5\baselineskip}{0.5\baselineskip} % Spacing around section titles, the order is: left, before and after

%------------------------------------------------
% Subsection

\titleformat
{\subsection} % Section type being modified
[block] % Shape type, can be: hang, block, display, runin, leftmargin, rightmargin, drop, wrap, frame
{\itshape} % Format of the whole section
{(\arabic{subsection})} % Format of the section label
{4pt} % Space between the title and label
{} % Code before the label

\titlespacing{\subsection}{0pt}{0.5\baselineskip}{0.5\baselineskip} % Spacing around section titles, the order is: left, before and after

\renewcommand\thesubsection{(\arabic{subsection})}

%----------------------------------------------------------------------------------------
%	CUSTOM QUESTION COMMANDS/ENVIRONMENTS
%----------------------------------------------------------------------------------------



% Command to print an assignment section title to split an assignment into major parts
\newcommand{\assignmentSection}[1]{
    \newpage
    {
        \centering % Centre the section title
        \vspace{2\baselineskip} % Whitespace before the entire section title

        \rule{0.8\textwidth}{0.5pt} % Horizontal rule

        \vspace{0.75\baselineskip} % Whitespace before the section title
        {\LARGE \textsc{#1}} % Section title, forced to be uppercase

        \rule{0.8\textwidth}{0.5pt} % Horizontal rule

        \vspace{\baselineskip} % Whitespace after the entire section title
    }
    \setcounter{section}{0}

}

%----------------------------------------------------------------------------------------
%	TITLE PAGE
%----------------------------------------------------------------------------------------

\author{\textbf{\assignmentNo\ \assignmentAuthorName}} % Set the default title page author field
\date{} % Don't use the default title page date field

\title{
    \thispagestyle{empty} % Suppress headers and footers
    \vspace{0.2\textheight} % Whitespace before the title
    \textbf{\assignmentClass}\\[5pt]
    \texttt{\assignmentTitle}\\[-4pt]
    % \ifdef{\assignmentSubTitle}{\texttt{\assignmentSubTitle}}{}
    \ifdef{\assignmentDate}{\assignmentDate}{} % If a due date is supplied, output it
    \ifdef{\assignmentClassInstructor}{{\large \textit{\assignmentClassInstructor}}}{} % If an instructor is supplied, output it
    \vspace{0.32\textheight} % Whitespace before the author name
}


\newcommand{\assignmentQuestionName}{Question}
\newcommand{\assignmentClass}{计算方法B}
\newcommand{\assignmentTitle}{Homework\ \#11}
\newcommand{\assignmentDate}{2020.6.1}
\newcommand{\assignmentNo}{PB17000297}
\newcommand{\assignmentAuthorName}{罗晏宸}
\usepackage{xspace}

\begin{document}

\maketitle

\thispagestyle{empty}

\newpage

\begin{question}

    \questiontext
    {
        试推导如下 Runge-Kutta 公式的局部截断误差和精度。
        \begin{equation*}
            \left\{
            \begin{aligned}
                y_{n + 1} & = y_{n} + \frac{h}{4}(3k_{1} + k_{2}) \\
                k_{1}     & = f(x_{n},\, y_{n})                   \\
                k_{2}     & = f(x_{n} + 2h,\, y_{n} + 2hk_{1})
            \end{aligned}
            \right.
        \end{equation*}
    }
    \answer
    {
        将$k_{1}$与$k_{2}$代入$ y_{n + 1} = y_{n} + \displaystyle \frac{h}{4}(3k_{1} + k_{2})$,得到
        \begin{align*}
            y_{n + 1} & = y_{n} + \frac{h}{4}\left[3f(x_{n},\, y_{n}) + f(x_{n} + 2h,\, y_{n} + 2hf(x_{n},\, y_{n}))\right]            \\
                      & = y_{n} + h\left[\frac{3}{4}f(x_{n},\, y_{n}) + \frac{1}{4}f(x_{n} + 2h,\, y_{n} + 2hf(x_{n},\, y_{n}))\right]
        \end{align*}
        将此式在$x_{n}$处做 Taylor 展开,得到
        \begin{align*}
            y_{n + 1} =\  & y_{n} + h\left[\frac{3}{4}f(x_{n},\, y_{n}) + \frac{1}{4}f(x_{n} + 2h,\, y_{n} + 2hf(x_{n},\, y_{n}))\right] \\
            =\            & y_{n} + h\bigg\{\frac{3}{4}f(x_{n},\, y_{n}) + \frac{1}{4}\big[f(x_{n},\, y_{n})                             \\
                          & + 2hf'_x(x_{n},\, y_{n}) + 2hf(x_{n},\, y_{n})f'_y(x_{n},\, y_{n}) + O(h^2)\big]\bigg\}
        \end{align*}
        而依微分方程
        \begin{equation*}
            \left\{
            \begin{aligned}
                y'(x) & = f(x, y) \\
                y(a)  & = y_0
            \end{aligned}
            \right.
            \qquad a \leqslant x \leqslant b %\tag{1}\label{ODE}
        \end{equation*}
        $y(x_{n + 1})$在$x_{n}$处的展开式为
        \begin{align*}
            y(x_{n + 1}) & = y(x_{n}) + hy'(x_{n}) + \frac{h^2}{2!}y''(x_{n}) + O(h^3)                                                                        \\
                         & = y(x_{n}) + hf(x_{n},\, y_{n}) + \frac{h^2}{2!}\left[f'_x(x_{n},\, y_{n}) + f'_y(x_{n},\, y_{n})f(x_{n},\, y_{n})\right] + O(h^3)
        \end{align*}
        所以局部截断误差
        \begin{align*}
            T_{n + 1} =\  & y(x_{n + 1}) - y_{n + 1}                                                                                                               \\
            =\            & hf(x_{n},\, y_{n}) + \frac{h^2}{2!}\left[f'_x(x_{n},\, y_{n}) + f'_y(x_{n},\, y_{n})f(x_{n},\, y_{n})\right] + O(h^3)                  \\
                          & - h\bigg\{\frac{3}{4}f(x_{n},\, y_{n}) + \frac{1}{4}\big[f(x_{n},\, y_{n})                                                             \\
                          & + 2hf'_x(x_{n},\, y_{n}) + 2hf(x_{n},\, y_{n})f'_y(x_{n},\, y_{n}) + O(h^2)\big]\bigg\}                                                \\
            =\            & \left(1 - \frac{3}{4} - \frac{1}{4}\right)hf(x_{n},\, y_{n}) + \left(\frac{1}{2!} - \frac{1}{4} \times 2\right)h^2f'_x(x_{n},\, y_{n}) \\
                          & + h^2\left(\frac{1}{2!} - \frac{1}{4} \times 2\right)f(x_{n},\, y_{n})f'_y(x_{n},\, y_{n}) + \left[O(h^3) - \frac{h}{4}O(h^2)\right]   \\
            =\            & O(h^3)
        \end{align*}
        所以题设公式是 2 阶的,即具有 2 阶精度。

        事实上由题设公式可知,公式主体部分有如下的形式
        \begin{equation*}
            c_1f(x_{n},\, y_{n}) + c_2f(x_{n} + ah,\, y_{n} + bhf(x_{n},\, y_{n}))
        \end{equation*}
        其中$c_1 = \dfrac{3}{4},\ c_2 = \dfrac{1}{4},\, a = 2,\, b = 2$,满足
        \begin{equation*}
            \left\{
            \begin{aligned}
                 & c_1 + c_2 = 1 \\
                 & 2c_2a = 1     \\
                 & 2c_2b = 1
            \end{aligned}
            \right.
        \end{equation*}
        因此题设公式是二阶 Runge-Kutta 公式
    }
\end{question}

\begin{question}
    \questiontext
    {
        讨论梯形格式 $y_{n + 1} = y_{n} + \displaystyle \frac{h}{2}\left[f(x_{n},\, y_{n}) + f(y_{n + 1},\, y_{n + 1})\right]$ 的绝对稳定性($h > 0$)。
    }
    \answer
    {

    }
\end{question}

\end{document}
