\documentclass[11pt]{article}
\usepackage[UTF8]{ctex}

%%%%%%%%%%%%%%%%%%%%%%%%%%%%%%%%%%%%%%%%%
% Cleese Assignment
% Structure Specification File
% Version 1.0 (27/5/2018)
%
% This template originates from:
% http://www.LaTeXTemplates.com
%
% Author:
% Vel (vel@LaTeXTemplates.com)
%
% License:
% CC BY-NC-SA 3.0 (http://creativecommons.org/licenses/by-nc-sa/3.0/)
%
%%%%%%%%%%%%%%%%%%%%%%%%%%%%%%%%%%%%%%%%%

%----------------------------------------------------------------------------------------
%	PACKAGES AND OTHER DOCUMENT CONFIGURATIONS
%----------------------------------------------------------------------------------------

\usepackage{lastpage} % Required to determine the last page number for the footer

\usepackage{graphicx} % Required to insert images

\setlength\parindent{0pt} % Removes all indentation from paragraphs

\usepackage[most]{tcolorbox} % Required for boxes that split across pages

\usepackage{booktabs} % Required for better horizontal rules in tables

\usepackage{listings} % Required for insertion of code

\usepackage{etoolbox} % Required for if statements

\usepackage{amsmath}
\usepackage{amsthm}
\usepackage{amssymb}
\usepackage{indentfirst}
\usepackage{diagbox}
\usepackage{subfigure}
\usepackage{float}
\usepackage{xcolor}
\usepackage[colorlinks, linkcolor = black]{hyperref}

\usepackage{enumerate}
\usepackage{enumitem}
\setlist{
    leftmargin = .1\linewidth,
    % rightmargin = .1\linewidth,
    % label=\emph{\alph*}.
}

\setlength{\parindent}{2em}

\usepackage{siunitx}
\sisetup
{
    output-exponent-marker = \ensuremath{\mathrm{E}},
    exponent-product = {},
    retain-explicit-plus,
    retain-zero-exponent,
}
%----------------------------------------------------------------------------------------
%	MARGINS
%----------------------------------------------------------------------------------------

\usepackage{geometry} % Required for adjusting page dimensions and margins

\geometry{
    paper=a4paper, % Change to letterpaper for US letter
    top=3cm, % Top margin
    bottom=3cm, % Bottom margin
    left=2.5cm, % Left margin
    right=2.5cm, % Right margin
    headheight=14pt, % Header height
    footskip=1.4cm, % Space from the bottom margin to the baseline of the footer
    headsep=1.2cm, % Space from the top margin to the baseline of the header
    %showframe, % Uncomment to show how the type block is set on the page
}

%----------------------------------------------------------------------------------------
%	FONT
%----------------------------------------------------------------------------------------

\usepackage[utf8]{inputenc} % Required for inputting international characters
\usepackage[T1]{fontenc} % Output font encoding for international characters

% \usepackage[sfdefault,light]{roboto} % Use the Roboto font

%----------------------------------------------------------------------------------------
%	HEADERS AND FOOTERS
%----------------------------------------------------------------------------------------

\usepackage{fancyhdr} % Required for customising headers and footers

\pagestyle{fancy} % Enable custom headers and footers

\lhead{\small\assignmentClass} % Left header; output the instructor in brackets if one was set
\chead{\small\assignmentTitle} % Centre header
\rhead{\small\ifdef{\assignmentAuthorName}{\assignmentAuthorName}{\ifdef{\assignmentDate}{Due\ \assignmentDate}{}}} % Right header; output the author name if one was set, otherwise the due date if that was set

\lfoot{} % Left footer
\cfoot{\small Page\ \thepage\ of\ \pageref{LastPage}} % Centre footer
\rfoot{} % Right footer

\renewcommand\headrulewidth{0.5pt} % Thickness of the header rule

%----------------------------------------------------------------------------------------
%	MODIFY SECTION STYLES
%----------------------------------------------------------------------------------------

\usepackage{titlesec} % Required for modifying sections

%------------------------------------------------
% Section

\titleformat
{\section} % Section type being modified
[block] % Shape type, can be: hang, block, display, runin, leftmargin, rightmargin, drop, wrap, frame
{\Large\bfseries} % Format of the whole section
{\arabic{section}} % Format of the section label
{6pt} % Space between the title and label
{} % Code before the label

\titlespacing{\section}{0pt}{0.5\baselineskip}{0.5\baselineskip} % Spacing around section titles, the order is: left, before and after

%------------------------------------------------
% Subsection

\titleformat
{\subsection} % Section type being modified
[block] % Shape type, can be: hang, block, display, runin, leftmargin, rightmargin, drop, wrap, frame
{\itshape} % Format of the whole section
{(\arabic{subsection})} % Format of the section label
{4pt} % Space between the title and label
{} % Code before the label

\titlespacing{\subsection}{0pt}{0.5\baselineskip}{0.5\baselineskip} % Spacing around section titles, the order is: left, before and after

\renewcommand\thesubsection{(\arabic{subsection})}

%----------------------------------------------------------------------------------------
%	CUSTOM QUESTION COMMANDS/ENVIRONMENTS
%----------------------------------------------------------------------------------------



% Command to print an assignment section title to split an assignment into major parts
\newcommand{\assignmentSection}[1]{
    \newpage
    {
        \centering % Centre the section title
        \vspace{2\baselineskip} % Whitespace before the entire section title

        \rule{0.8\textwidth}{0.5pt} % Horizontal rule

        \vspace{0.75\baselineskip} % Whitespace before the section title
        {\LARGE \textsc{#1}} % Section title, forced to be uppercase

        \rule{0.8\textwidth}{0.5pt} % Horizontal rule

        \vspace{\baselineskip} % Whitespace after the entire section title
    }
    \setcounter{section}{0}

}

%----------------------------------------------------------------------------------------
%	TITLE PAGE
%----------------------------------------------------------------------------------------

\author{\textbf{\assignmentNo\ \assignmentAuthorName}} % Set the default title page author field
\date{} % Don't use the default title page date field

\title{
    \thispagestyle{empty} % Suppress headers and footers
    \vspace{0.2\textheight} % Whitespace before the title
    \textbf{\assignmentClass}\\[5pt]
    \texttt{\assignmentTitle}\\[-4pt]
    % \ifdef{\assignmentSubTitle}{\texttt{\assignmentSubTitle}}{}
    \ifdef{\assignmentDate}{\assignmentDate}{} % If a due date is supplied, output it
    \ifdef{\assignmentClassInstructor}{{\large \textit{\assignmentClassInstructor}}}{} % If an instructor is supplied, output it
    \vspace{0.32\textheight} % Whitespace before the author name
}
 % Include the file specifying the document structure and custom commands


\newcommand{\assignmentQuestionName}{Question} % The word to be used as a prefix to question numbers; example alternatives: Problem, Exercise
\newcommand{\assignmentClass}{计算方法B} % Course/class
\newcommand{\assignmentTitle}{Homework\ \#1} % Assignment title or name
\newcommand{\assignmentDate}{2020.3.27} % date
\newcommand{\assignmentNo}{PB17000297}
\newcommand{\assignmentAuthorName}{罗晏宸} % Student name

\begin{document}

\maketitle % Print the title page

\thispagestyle{empty} % Suppress headers and footers on the title page

\newpage

\begin{question}

	\questiontext{阅读绪论并给出计算如下函数的可靠数值计算方法,
		使其尽量达到更好的精度。\\ 其中,(1)、(2)中$x$很靠近$0$且$a > 0$;(3)中$x \gg a$}

	\begin{subquestion}{$f(x) = (a + x)^n - a^n$}
		\answer{
			\begin{align*}
				f(x) & = (a + x)^n - a^n                                                           \\
				     & = \sum_{i = 0}^n{\binom{n}{i} a^{n - i}x^i} - a^n                           \\
				     & = \sum_{i = 1}^n{\binom{n}{i} a^{n - i}x^i}                                 \\
				     & = \underbrace{(\cdots(}_{n\text{个}(}x + a)x + a^2)x + \cdots + a^{n - 1})x
			\end{align*}
		}
	\end{subquestion}
	\begin{subquestion}{$f(x) = \cos{(a + x)} - \cos{a}$}
		\answer{
			\begin{align*}
				f(x) & = \cos{(a + x)} - \cos{a}                                    \\
				     & = - 2 \sin{\frac{(a + x) + a}{2}}\sin{\frac{(a + x) - a}{2}} \\
				     & = - 2 \sin{(a + \frac{x}{2})}\sin{\frac{x}{2}}
			\end{align*}
		}
	\end{subquestion}
	\begin{subquestion}{$f(x) = x - \sqrt{x^2 + a}$}
		\answer{
			\begin{align*}
				f(x) & = x - \sqrt{x^2 + a}                                                  \\
				     & = \frac{(x + \sqrt{x^2 + a})(x - \sqrt{x^2 + a})}{x + \sqrt{x^2 + a}} \\
				     & = - \frac{a}{x + \sqrt{x^2 + a}}
			\end{align*}
		}
	\end{subquestion}

\end{question}

\begin{question}

	\questiontext{设有精确值$x^* = 0.0202005$,则其近似值$x = 0.020200$有几位有效数字?近似值$x$的绝对误差是多少?}

	\answer{
		近似值有5位有效数字,绝对误差为:
		\begin{equation*}
			x^* - x = 0.0202005 - 0.020200 = 0.0000005
		\end{equation*}
	}

\end{question}

\begin{question}

	\questiontext{
		设有插值节点$a \leqslant x_0 < x_1 < \cdots < x_n \leqslant b$证明与这些节点相应的Lagrange插值基函数$$\{l_i(x),\quad i = 0,\, 1,\, \cdots,\, n\}$$是线性无关的。
	}
	\answer{
		假设存在$n$个数$c_0,\, c_1,\, \cdots,\, c_n$使得
		\begin{equation*}
			\sum_{i = 0}^n{c_il_i(x)} \equiv 0
		\end{equation*}
		成立,则对于$x = x_0$,有:
		\begin{align*}
			 &             & \sum_{i = 0}^n{c_il_i(x_0)}                 & = 0 & \\
			 & \Rightarrow & c_0l_0(x_0) + \sum_{i = 1}^n{c_il_i(x_0)}   & = 0 & \\
			 & \Rightarrow & c_0 \times 1 + \sum_{i = 1}^n{c_i \times 0} & = 0 & \\
			 & \Rightarrow & c_0                                         & = 0 &
		\end{align*}
		事实上,对于任意的$x = x_j,\quad 0 \leqslant j \leqslant n$,都有:
		\begin{align*}
			 &             & \sum_{i = 0}^n{c_il_i(x_0)}                          & = 0 & \\
			 & \Rightarrow & c_jl_j(x_j) + \sum_{i = 0,i \neq j}^n{c_il_i(x_0)}   & = 0 & \\
			 & \Rightarrow & c_j \times 1 + \sum_{i = 0,i \neq j}^n{c_i \times 0} & = 0 & \\
			 & \Rightarrow & c_j                                                  & = 0 &
		\end{align*}
		即$c_0 = c_1 = \cdots = c_n = 0$,因此$\{l_i(x),\quad i = 0,\, 1,\, \cdots,\, n\}$是线性无关的。
	}

\end{question}

\begin{question}

	\questiontext{利用插值数据$(-1.0,\,0.0)$, $(1.0,\,1.0)$, $(4.0,\,2.0)$, $(5.0,\,4.0)$,构造出三次 Lagrange 插值多项式$L_3(x)$,并计算$L_3(2.0)$, $L_3(4.0)$。}


	\answer{
		设$L_3(x) = ax^3 + bx^2 + cx + d$,代入插值数据,得到
		\begin{align*}
			            & \left\{
			\begin{aligned}
				-   & a + &    & b - &   & c + d = 0 \\
				    & a + &    & b + &   & c + d = 1 \\
				64  & a + & 16 & b + & 4 & c + d = 2 \\
				125 & a + & 25 & b + & 5 & c + d = 4
			\end{aligned}
			\right.               \\
			\Rightarrow &
			\left\{
			\begin{aligned}
				a & = \frac{3}{40}  \\
				b & = -\frac{1}{3}  \\
				c & = \frac{17}{40} \\
				d & = \frac{5}{6}
			\end{aligned}
			\right.
		\end{align*}
		即
		\begin{equation*}
			L_3(x) = \frac{3}{40}x^3 - \frac{1}{3}x^2 + \frac{17}{40}x + \frac{5}{6}
		\end{equation*}
		进而
		\begin{align*}
			L_3(2.0) & = \frac{19}{20} \\
			L_3(4.0) & = 2
		\end{align*}
	}

\end{question}

\end{document}
