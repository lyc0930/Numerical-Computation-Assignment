\documentclass[11pt]{article}
\usepackage[UTF8]{ctex}

%%%%%%%%%%%%%%%%%%%%%%%%%%%%%%%%%%%%%%%%%
% Cleese Assignment
% Structure Specification File
% Version 1.0 (27/5/2018)
%
% This template originates from:
% http://www.LaTeXTemplates.com
%
% Author:
% Vel (vel@LaTeXTemplates.com)
%
% License:
% CC BY-NC-SA 3.0 (http://creativecommons.org/licenses/by-nc-sa/3.0/)
%
%%%%%%%%%%%%%%%%%%%%%%%%%%%%%%%%%%%%%%%%%

%----------------------------------------------------------------------------------------
%	PACKAGES AND OTHER DOCUMENT CONFIGURATIONS
%----------------------------------------------------------------------------------------

\usepackage{lastpage} % Required to determine the last page number for the footer

\usepackage{graphicx} % Required to insert images

\setlength\parindent{0pt} % Removes all indentation from paragraphs

\usepackage[most]{tcolorbox} % Required for boxes that split across pages

\usepackage{booktabs} % Required for better horizontal rules in tables

\usepackage{listings} % Required for insertion of code

\usepackage{etoolbox} % Required for if statements

\usepackage{amsmath}
\usepackage{amsthm}
\usepackage{amssymb}
\usepackage{indentfirst}
\usepackage{diagbox}
\usepackage{subfigure}
\usepackage{float}
\usepackage{xcolor}
\usepackage[colorlinks, linkcolor = black]{hyperref}

\usepackage{enumerate}
\usepackage{enumitem}
\setlist{
    leftmargin = .1\linewidth,
    % rightmargin = .1\linewidth,
    % label=\emph{\alph*}.
}

\setlength{\parindent}{2em}

\usepackage{siunitx}
\sisetup
{
    output-exponent-marker = \ensuremath{\mathrm{E}},
    exponent-product = {},
    retain-explicit-plus,
    retain-zero-exponent,
}
%----------------------------------------------------------------------------------------
%	MARGINS
%----------------------------------------------------------------------------------------

\usepackage{geometry} % Required for adjusting page dimensions and margins

\geometry{
    paper=a4paper, % Change to letterpaper for US letter
    top=3cm, % Top margin
    bottom=3cm, % Bottom margin
    left=2.5cm, % Left margin
    right=2.5cm, % Right margin
    headheight=14pt, % Header height
    footskip=1.4cm, % Space from the bottom margin to the baseline of the footer
    headsep=1.2cm, % Space from the top margin to the baseline of the header
    %showframe, % Uncomment to show how the type block is set on the page
}

%----------------------------------------------------------------------------------------
%	FONT
%----------------------------------------------------------------------------------------

\usepackage[utf8]{inputenc} % Required for inputting international characters
\usepackage[T1]{fontenc} % Output font encoding for international characters

% \usepackage[sfdefault,light]{roboto} % Use the Roboto font

%----------------------------------------------------------------------------------------
%	HEADERS AND FOOTERS
%----------------------------------------------------------------------------------------

\usepackage{fancyhdr} % Required for customising headers and footers

\pagestyle{fancy} % Enable custom headers and footers

\lhead{\small\assignmentClass} % Left header; output the instructor in brackets if one was set
\chead{\small\assignmentTitle} % Centre header
\rhead{\small\ifdef{\assignmentAuthorName}{\assignmentAuthorName}{\ifdef{\assignmentDate}{Due\ \assignmentDate}{}}} % Right header; output the author name if one was set, otherwise the due date if that was set

\lfoot{} % Left footer
\cfoot{\small Page\ \thepage\ of\ \pageref{LastPage}} % Centre footer
\rfoot{} % Right footer

\renewcommand\headrulewidth{0.5pt} % Thickness of the header rule

%----------------------------------------------------------------------------------------
%	MODIFY SECTION STYLES
%----------------------------------------------------------------------------------------

\usepackage{titlesec} % Required for modifying sections

%------------------------------------------------
% Section

\titleformat
{\section} % Section type being modified
[block] % Shape type, can be: hang, block, display, runin, leftmargin, rightmargin, drop, wrap, frame
{\Large\bfseries} % Format of the whole section
{\arabic{section}} % Format of the section label
{6pt} % Space between the title and label
{} % Code before the label

\titlespacing{\section}{0pt}{0.5\baselineskip}{0.5\baselineskip} % Spacing around section titles, the order is: left, before and after

%------------------------------------------------
% Subsection

\titleformat
{\subsection} % Section type being modified
[block] % Shape type, can be: hang, block, display, runin, leftmargin, rightmargin, drop, wrap, frame
{\itshape} % Format of the whole section
{(\arabic{subsection})} % Format of the section label
{4pt} % Space between the title and label
{} % Code before the label

\titlespacing{\subsection}{0pt}{0.5\baselineskip}{0.5\baselineskip} % Spacing around section titles, the order is: left, before and after

\renewcommand\thesubsection{(\arabic{subsection})}

%----------------------------------------------------------------------------------------
%	CUSTOM QUESTION COMMANDS/ENVIRONMENTS
%----------------------------------------------------------------------------------------



% Command to print an assignment section title to split an assignment into major parts
\newcommand{\assignmentSection}[1]{
    \newpage
    {
        \centering % Centre the section title
        \vspace{2\baselineskip} % Whitespace before the entire section title

        \rule{0.8\textwidth}{0.5pt} % Horizontal rule

        \vspace{0.75\baselineskip} % Whitespace before the section title
        {\LARGE \textsc{#1}} % Section title, forced to be uppercase

        \rule{0.8\textwidth}{0.5pt} % Horizontal rule

        \vspace{\baselineskip} % Whitespace after the entire section title
    }
    \setcounter{section}{0}

}

%----------------------------------------------------------------------------------------
%	TITLE PAGE
%----------------------------------------------------------------------------------------

\author{\textbf{\assignmentNo\ \assignmentAuthorName}} % Set the default title page author field
\date{} % Don't use the default title page date field

\title{
    \thispagestyle{empty} % Suppress headers and footers
    \vspace{0.2\textheight} % Whitespace before the title
    \textbf{\assignmentClass}\\[5pt]
    \texttt{\assignmentTitle}\\[-4pt]
    % \ifdef{\assignmentSubTitle}{\texttt{\assignmentSubTitle}}{}
    \ifdef{\assignmentDate}{\assignmentDate}{} % If a due date is supplied, output it
    \ifdef{\assignmentClassInstructor}{{\large \textit{\assignmentClassInstructor}}}{} % If an instructor is supplied, output it
    \vspace{0.32\textheight} % Whitespace before the author name
}


\newcommand{\assignmentQuestionName}{Question}
\newcommand{\assignmentClass}{计算方法B}
\newcommand{\assignmentTitle}{Homework\ \#9}
\newcommand{\assignmentDate}{2020.5.23}
\newcommand{\assignmentNo}{PB17000297}
\newcommand{\assignmentAuthorName}{罗晏宸}
\usepackage{xspace}

% Add a period to the end of an abbreviation unless there's one
% already, then \xspace.
\makeatletter
\DeclareRobustCommand\onedot{\futurelet\@let@token\@onedot}
\def\@onedot{\ifx\@let@token.\else.\null\fi\xspace}
\def\ie{\emph{i.e}\onedot}
\makeatother

\begin{document}

\maketitle

\thispagestyle{empty}

\newpage

\begin{question}

    \questiontext
    {
        给定函数$f(x)$的离散值表,
        \begin{table}[h]
            \centering
            \begin{tabular}{|c|c|c|c|c|}
                \hline
                $x$    & $0.00$ & $0.02$ & $0.04$ & $0.06$ \\ \hline
                $f(x)$ & $6.0$  & $4.0$  & $2.0$  & $8.0$  \\ \hline
            \end{tabular}
        \end{table}
        分别用向前、向后及中心差商公式计算$f'(0.02)$,$f'(0.04)$。
    }
    \answer
    {
        由一阶向前差商公式
        \begin{equation*}
            f'(x_{n}) \approx \frac{f(x_{n + 1}) - f(x_{n})}{x_{n + 1} - x_{n}}
        \end{equation*}
        可得
        \begin{align*}
            f'(0.02) & \approx \frac{f(0.04) - f(0.02)}{0.04 - 0.02} \\
                     & = \frac{2.0 - 4.0}{0.02}                      \\
                     & = -100                                        \\
            \\
            f'(0.04) & \approx \frac{f(0.06) - f(0.04)}{0.06 - 0.04} \\
                     & = \frac{8.0 - 2.0}{0.02}                      \\
                     & = 300
        \end{align*}
        由一阶向后差商公式
        \begin{equation*}
            f'(x_{n + 1}) \approx \frac{f(x_{n + 1}) - f(x_{n})}{x_{n + 1} - x_{n}}
        \end{equation*}
        可得
        \begin{align*}
            f'(0.02) & \approx \frac{f(0.02) - f(0.00)}{0.02 - 0.00} \\
                     & = \frac{4.0 - 6.0}{0.02}                      \\
                     & = -100                                        \\
            \\
            f'(0.04) & \approx \frac{f(0.04) - f(0.02)}{0.04 - 0.02} \\
                     & = \frac{2.0 - 4.0}{0.02}                      \\
                     & = -100
        \end{align*}
        由一阶中心差商公式
        \begin{equation*}
            f'(x_n) \approx \frac{f(x_{n + 1}) - f(x_{n - 1})}{x_{n + 1} - x_{n - 1}}
        \end{equation*}
        可得
        \begin{align*}
            f'(0.02) & \approx \frac{f(0.04) - f(0.00)}{0.04 - 0.00} \\
                     & = \frac{2.0 - 6.0}{0.04}                      \\
                     & = -100                                        \\
            \\
            f'(0.04) & \approx \frac{f(0.06) - f(0.02)}{0.06 - 0.02} \\
                     & = \frac{8.0 - 4.0}{0.04}                      \\
                     & = 100
        \end{align*}
    }
\end{question}

\begin{question}
    \questiontext
    {
        求积分$\displaystyle \int_{0}^{1}\!x^2f(x)\,\text{d}x$的 2 点 Gauss 积分公式,这里 $x^2$ 为权重函数。
    }
    \answer
    {
        利用 Gram-Schmidt 正交化
        \begin{equation*}
            \left\{
            \begin{aligned}
                p_0(x) & = f_0(x)                                                                                                                                                         \\
                p_1(x) & = f_1(x) - \frac{\left(f_1(x),\, p_0(x)\right)}{\left(p_0(x),\, p_0(x)\right)}p_0(x)                                                                             \\
                p_2(x) & = f_2(x) - \frac{\left(f_2(x),\, p_0(x)\right)}{\left(p_0(x),\, p_0(x)\right)}p_0(x) - \frac{\left(f_2(x),\, p_1(x)\right)}{\left(p_1(x),\, p_1(x)\right)}p_1(x)
            \end{aligned}
            \right.
        \end{equation*}
        可由$\mathbb{P}_n$的一组基$\{1,\, x,\, \cdots,\, x^n\}$依次求出正交多项式序列:
        \begin{equation*}
            \left\{
            \begin{aligned}
                p_0(x) & = 1                                                                                                                                                                                                                                                                                                                                                                                                       \\
                p_1(x) & = x - \frac{\displaystyle \int_{0}^{1}\!x^2 \cdot x \cdot 1 \,\text{d}x}{\displaystyle \int_{0}^{1}\!x^2 \cdot 1 \cdot 1 \,\text{d}x} \cdot 1 = x - \frac{3}{4}                                                                                                                                                                                                                                           \\
                p_2(x) & = x^2 - \frac{\displaystyle \int_{0}^{1}\!x^2 \cdot x^2 \cdot 1 \,\text{d}x}{\displaystyle \int_{0}^{1}\!x^2 \cdot 1 \cdot 1 \,\text{d}x} \cdot 1 - \frac{\displaystyle \int_{0}^{1}\!x^2 \cdot x^2 \cdot \left(x - \frac{3}{4}\right) \,\text{d}x}{\displaystyle \int_{0}^{1}\!x^2 \cdot \left(x - \frac{3}{4}\right) \cdot \left(x - \frac{3}{4}\right) \,\text{d}x} \cdot \left(x - \frac{3}{4}\right) \\
                       & = x^2 - \frac{4}{3}x + \frac{2}{5}
            \end{aligned}
            \right.
        \end{equation*}
        $p_2(x)$的两个零点为$x_1 = \dfrac{10 - \sqrt{10}}{15},\, x_2 = \dfrac{10 + \sqrt{10}}{15}$,积分系数为
        \begin{align*}
            \alpha_1 & = \int_{0}^{1}\!x^2I_1(x)\,\text{d}x = \int_{0}^{1}\!x^2\frac{x - x_2}{x_1 - x_2}\,\text{d}x = \frac{8 - \sqrt{10}}{48} \\
            \alpha_2 & = \int_{0}^{1}\!x^2I_2(x)\,\text{d}x = \int_{0}^{1}\!x^2\frac{x - x_1}{x_2 - x_1}\,\text{d}x = \frac{8 + \sqrt{10}}{48}
        \end{align*}
        因此两点 Gauss 公式为
        \begin{align*}
            \int_{0}^{1}\!x^2f(x)\,\text{d}x \approx & G_2(f)                                                                                                                            \\
            =                                        & \sum_{i = 1}^{2}\alpha_if(x_i)                                                                                                    \\
            =                                        & \frac{8 - \sqrt{10}}{48}f\left(\frac{10 - \sqrt{10}}{15}\right) + \frac{8 + \sqrt{10}}{48}f\left(\frac{10 + \sqrt{10}}{15}\right)
        \end{align*}

    }
\end{question}

\begin{question}
    \questiontext
    {
        设有常微分初值问题
        \begin{equation*}
            \left\{
            \begin{aligned}
                y'(x) & = - y(x) ,\qquad (0 \leqslant x \leqslant 1) \\
                y(0)  & = 1
            \end{aligned}
            \right.
        \end{equation*}
        假设求解区间$[0, 1]$被 $n$ 等分($n$充分大),令$h = \dfrac{1}{n},\ x_k = \dfrac{k}{n}\ (k = 0,\, 1,\, \cdots,\, n)$
    }
    \begin{subquestion}{分别写出用向前 Euler 公式,向后 Euler 公式,梯形格式以及改进的 Euler 公式求上述微分方程数值解时的差分格式(即分别写出此四种方法/公式下,$y_{k + 1}$与$y_k$之间的递推关系式)}
        \answer
        {
            向前 Euler 公式
            \begin{equation*}
                y_{k + 1} = y_{k} + hf(x_{k},\, y_{k}) = y_{k} - \frac{1}{n}y_{k} = \frac{n - 1}{n}y_{k}
            \end{equation*}
            向后 Euler 公式
            \begin{align*}
                y_{k + 1}             & = y_{k} + hf(x_{k + 1},\, y_{k + 1}) = y_{k} - \frac{1}{n}y_{k + 1} \\
                \Rightarrow y_{k + 1} & = \frac{n}{n + 1}y_{k}
            \end{align*}
            梯形公式
            \begin{align*}
                y_{k + 1}             & = y_{k} + \frac{h}{2}\left(f(x_{k},\, y_{k}) + f(x_{k + 1},\, y_{k + 1})\right) = y_{k} - \frac{1}{2n}\left(y_{k} + y_{k + 1}\right) \\
                \Rightarrow y_{k + 1} & = \frac{2n - 1}{2n + 1}y_{k}
            \end{align*}
            改进的 Euler 公式
            \begin{align*}
                y_{k + 1}             & = y_{k} + \frac{h}{2}\left(f(x_{k},\, y_{k}) + f(x_{k + 1},\, y_{k} + hf(x_{k}, y_{k}))\right) = y_{k} - \frac{1}{2n}\left(y_{k} + \frac{n + 1}{n}y_{k}\right) \\
                \Rightarrow y_{k + 1} & = \frac{2n^2 - 2n - 1}{2n^2}y_{k}
            \end{align*}
        }
    \end{subquestion}
    \begin{subquestion}{设$y_0 = y(0)$,分别求此四种公式(方法)下的近似$y_n$的表达式(注:这里的$y_n$即是$y(x_n) \equiv y(1)$的近似值)}
        \answer
        {
            由$y_0 = y(0) = 1$,由向前 Euler 公式下递推式可得
            \begin{align*}
                y_{n} = \frac{n - 1}{n}y_{n - 1} = \cdots = \left(\frac{n - 1}{n}\right)^{n}y_{0} = \left(\frac{n - 1}{n}\right)^{n}
            \end{align*}
            同理对于向后 Euler 公式
            \begin{align*}
                y_{n} = \left(\frac{n}{n + 1}\right)^{n}
            \end{align*}
            梯形公式
            \begin{align*}
                y_{n} = \left(\frac{2n - 1}{2n + 1}\right)^{n}
            \end{align*}
            改进的 Euler 公式
            \begin{align*}
                y_{n} = \left(\frac{2n^2 - 2n - 1}{2n^2}\right)^{n}
            \end{align*}
        }
    \end{subquestion}
    \begin{subquestion}{当$n$充分大(即区间长度$h \rightarrow 0$)时,分别判断四种方法下的近似值$y_n$是否收敛到原问题的真解$y(x)$在$x = 1$处的值(\ie $y(1)$)}
        \answer
        {

            可由
            \begin{align*}
                y'(x) = -y(x)                                                 \\
                \Rightarrow \frac{\text{d}y}{\text{d}x} = -y                  \\
                \Rightarrow \frac{1}{y}\text{d}y = -\text{d}x                 \\
                \Rightarrow \int\!\frac{1}{y}\,\text{d}y = \int\!-\,\text{d}x \\
                \Rightarrow \ln y = -x + C'                                   \\
                \Rightarrow y = e^{-x} + C
            \end{align*}
            代入$y(0) = 1$可得$C = 0$,即原问题
            \begin{equation*}
                \left\{
                \begin{aligned}
                    y'(x) & = - y(x) ,\qquad (0 \leqslant x \leqslant 1) \\
                    y(0)  & = 1
                \end{aligned}
                \right.
            \end{equation*}
            的解为
            \begin{equation*}
                y(x) = e^{-x}
            \end{equation*}

            其在$x = 1$处的值$y(1) = e^{-1}$。当$n$充分大(即区间长度$h \rightarrow 0$)时,有
            \begin{align*}
                 & \lim_{n \rightarrow \infty}\left(\frac{n - 1}{n}\right)^{n} = \lim_{n \rightarrow \infty}\left(1 - \frac{1}{n}\right)\frac{1}{\left(1 + \dfrac{1}{n - 1}\right)^{n - 1}} = \frac{1}{e}                            \\
                 & \lim_{n \rightarrow \infty}\left(\frac{n}{n + 1}\right)^{n} = \lim_{n \rightarrow \infty}\frac{1}{\left(1 + \dfrac{1}{n}\right)^{n}} = \frac{1}{e}                                                                \\
                 & \lim_{n \rightarrow \infty}\left(\frac{2n - 1}{2n + 1}\right)^{n} = \lim_{n \rightarrow \infty}\sqrt{\frac{2n - 1}{2n + 1}}\frac{1}{\left(1 + \dfrac{1}{n - \frac{1}{2}}\right)^{n - \frac{1}{2}}} =  \frac{1}{e} \\
                 & \lim_{n \rightarrow \infty}\left(\frac{2n^2 - 2n - 1}{2n^2}\right)^{n} = \frac{1}{e}
            \end{align*}
            其中最后一个极限可由$\left(\displaystyle \frac{n - 2}{n - 1}\right)^{n} \leqslant \left(\displaystyle \frac{2n^2 - 2n - 1}{2n^2}\right)^{n} \leqslant \left(\displaystyle \frac{n - 1}{n}\right)^{n}$,进而由夹逼定理得到
            因此四种方法都收敛到真解在$x = 1$处的值
        }
    \end{subquestion}
\end{question}
\end{document}
