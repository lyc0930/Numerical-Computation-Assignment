\documentclass[11pt]{article}
\usepackage[UTF8]{ctex}

%%%%%%%%%%%%%%%%%%%%%%%%%%%%%%%%%%%%%%%%%
% Cleese Assignment
% Structure Specification File
% Version 1.0 (27/5/2018)
%
% This template originates from:
% http://www.LaTeXTemplates.com
%
% Author:
% Vel (vel@LaTeXTemplates.com)
%
% License:
% CC BY-NC-SA 3.0 (http://creativecommons.org/licenses/by-nc-sa/3.0/)
%
%%%%%%%%%%%%%%%%%%%%%%%%%%%%%%%%%%%%%%%%%

%----------------------------------------------------------------------------------------
%	PACKAGES AND OTHER DOCUMENT CONFIGURATIONS
%----------------------------------------------------------------------------------------

\usepackage{lastpage} % Required to determine the last page number for the footer

\usepackage{graphicx} % Required to insert images

\setlength\parindent{0pt} % Removes all indentation from paragraphs

\usepackage[most]{tcolorbox} % Required for boxes that split across pages

\usepackage{booktabs} % Required for better horizontal rules in tables

\usepackage{listings} % Required for insertion of code

\usepackage{etoolbox} % Required for if statements

\usepackage{amsmath}
\usepackage{amsthm}
\usepackage{amssymb}
\usepackage{indentfirst}
\usepackage{diagbox}
\usepackage{subfigure}
\usepackage{float}
\usepackage{xcolor}
\usepackage[colorlinks, linkcolor = black]{hyperref}

\usepackage{enumerate}
\usepackage{enumitem}
\setlist{
    leftmargin = .1\linewidth,
    % rightmargin = .1\linewidth,
    % label=\emph{\alph*}.
}

\setlength{\parindent}{2em}

\usepackage{siunitx}
\sisetup
{
    output-exponent-marker = \ensuremath{\mathrm{E}},
    exponent-product = {},
    retain-explicit-plus,
    retain-zero-exponent,
}
%----------------------------------------------------------------------------------------
%	MARGINS
%----------------------------------------------------------------------------------------

\usepackage{geometry} % Required for adjusting page dimensions and margins

\geometry{
    paper=a4paper, % Change to letterpaper for US letter
    top=3cm, % Top margin
    bottom=3cm, % Bottom margin
    left=2.5cm, % Left margin
    right=2.5cm, % Right margin
    headheight=14pt, % Header height
    footskip=1.4cm, % Space from the bottom margin to the baseline of the footer
    headsep=1.2cm, % Space from the top margin to the baseline of the header
    %showframe, % Uncomment to show how the type block is set on the page
}

%----------------------------------------------------------------------------------------
%	FONT
%----------------------------------------------------------------------------------------

\usepackage[utf8]{inputenc} % Required for inputting international characters
\usepackage[T1]{fontenc} % Output font encoding for international characters

% \usepackage[sfdefault,light]{roboto} % Use the Roboto font

%----------------------------------------------------------------------------------------
%	HEADERS AND FOOTERS
%----------------------------------------------------------------------------------------

\usepackage{fancyhdr} % Required for customising headers and footers

\pagestyle{fancy} % Enable custom headers and footers

\lhead{\small\assignmentClass} % Left header; output the instructor in brackets if one was set
\chead{\small\assignmentTitle} % Centre header
\rhead{\small\ifdef{\assignmentAuthorName}{\assignmentAuthorName}{\ifdef{\assignmentDate}{Due\ \assignmentDate}{}}} % Right header; output the author name if one was set, otherwise the due date if that was set

\lfoot{} % Left footer
\cfoot{\small Page\ \thepage\ of\ \pageref{LastPage}} % Centre footer
\rfoot{} % Right footer

\renewcommand\headrulewidth{0.5pt} % Thickness of the header rule

%----------------------------------------------------------------------------------------
%	MODIFY SECTION STYLES
%----------------------------------------------------------------------------------------

\usepackage{titlesec} % Required for modifying sections

%------------------------------------------------
% Section

\titleformat
{\section} % Section type being modified
[block] % Shape type, can be: hang, block, display, runin, leftmargin, rightmargin, drop, wrap, frame
{\Large\bfseries} % Format of the whole section
{\arabic{section}} % Format of the section label
{6pt} % Space between the title and label
{} % Code before the label

\titlespacing{\section}{0pt}{0.5\baselineskip}{0.5\baselineskip} % Spacing around section titles, the order is: left, before and after

%------------------------------------------------
% Subsection

\titleformat
{\subsection} % Section type being modified
[block] % Shape type, can be: hang, block, display, runin, leftmargin, rightmargin, drop, wrap, frame
{\itshape} % Format of the whole section
{(\arabic{subsection})} % Format of the section label
{4pt} % Space between the title and label
{} % Code before the label

\titlespacing{\subsection}{0pt}{0.5\baselineskip}{0.5\baselineskip} % Spacing around section titles, the order is: left, before and after

\renewcommand\thesubsection{(\arabic{subsection})}

%----------------------------------------------------------------------------------------
%	CUSTOM QUESTION COMMANDS/ENVIRONMENTS
%----------------------------------------------------------------------------------------



% Command to print an assignment section title to split an assignment into major parts
\newcommand{\assignmentSection}[1]{
    \newpage
    {
        \centering % Centre the section title
        \vspace{2\baselineskip} % Whitespace before the entire section title

        \rule{0.8\textwidth}{0.5pt} % Horizontal rule

        \vspace{0.75\baselineskip} % Whitespace before the section title
        {\LARGE \textsc{#1}} % Section title, forced to be uppercase

        \rule{0.8\textwidth}{0.5pt} % Horizontal rule

        \vspace{\baselineskip} % Whitespace after the entire section title
    }
    \setcounter{section}{0}

}

%----------------------------------------------------------------------------------------
%	TITLE PAGE
%----------------------------------------------------------------------------------------

\author{\textbf{\assignmentNo\ \assignmentAuthorName}} % Set the default title page author field
\date{} % Don't use the default title page date field

\title{
    \thispagestyle{empty} % Suppress headers and footers
    \vspace{0.2\textheight} % Whitespace before the title
    \textbf{\assignmentClass}\\[5pt]
    \texttt{\assignmentTitle}\\[-4pt]
    % \ifdef{\assignmentSubTitle}{\texttt{\assignmentSubTitle}}{}
    \ifdef{\assignmentDate}{\assignmentDate}{} % If a due date is supplied, output it
    \ifdef{\assignmentClassInstructor}{{\large \textit{\assignmentClassInstructor}}}{} % If an instructor is supplied, output it
    \vspace{0.32\textheight} % Whitespace before the author name
}


\newcommand{\assignmentQuestionName}{Question}
\newcommand{\assignmentClass}{计算方法B}
\newcommand{\assignmentTitle}{Homework\ \#2}
\newcommand{\assignmentDate}{2020.4.1}
\newcommand{\assignmentNo}{PB17000297}
\newcommand{\assignmentAuthorName}{罗晏宸}

\begin{document}

\maketitle

\thispagestyle{empty}

\newpage

\begin{question}

	\questiontext{$f(x) = \sqrt{x}$在离散点有$f(81) = 9$, $f(100) = 10$, $f(121) = 11$,用插值方法计算$\sqrt{108}$的近似值,根据误差公式给出误差界。}
	\answer{
		二次插值函数为
		\begin{align*}
			L_2(x) &= \frac{(x - 100)(x - 121)}{(81 - 100)(81 - 121)} f(81) + \frac{(x - 81)(x - 121)}{(100 - 81)(100 - 121)} f(100) \\
			         & + \frac{(x - 81)(x - 100)}{(121 - 81)(121 - 100)} f(121)                                                       \\
			&=         \frac{1}{7980}(-x^2 + 601 x + 29700)
		\end{align*}
		计算函数在$x = 108$处的近似值为
		\begin{align*}
			\sqrt{108} &= f(108) \approx L_2(108)                         \\
			&=             \frac{1}{7980}(-108^2 + 601 \times 108 + 29700) \\
			&=             \frac{6912}{665}                                \\
			&\approx       10.394
		\end{align*}
		二次插值函数的误差为
		\begin{align*}
			R_2(108) &=  \frac{f^{(3)}(\xi)}{3!}(108 - 81)(108 - 100)(108 - 121), & \xi \in [81, 121] \\
			&=           -\frac{351}{2}\xi^{-2.5},                                & \xi \in [81, 121]
		\end{align*}
		$R_2(x)$在区间上是非正单调递增的,有
		\begin{equation*}
			|R_2(108)| \leqslant \left|-\frac{351}{2}81^{-2.5}\right| = \frac{13}{4374} \approx 0.002972
		\end{equation*}
		即误差界约为$0.002972$
	}
\end{question}

\begin{question}

	\questiontext{利用下面的函数值表,做出差商表,写出相应的牛顿插值多项式,并计算$f(1.5)$的近似值
		\begin{table}[h]
			\centering
			\begin{tabular}{|c|c|c|c|c|}
				\hline
				$x$    & $1.0$ & $2.0$ & $3.0$ & $4.0$ \\ \hline
				$f(x)$ & $2.0$ & $4.0$ & $8.0$ & $5.0$ \\ \hline
			\end{tabular}
		\end{table}
	}

	\answer{
		计算给定数据的一至三阶差商:
		\begin{align*}
			f[1, 2]       & = \frac{4.0 - 2.0}{2 - 1} = 2                                         \\
			f[2, 3]       & = \frac{8.0 - 4.0}{3 - 2} = 4                                         \\
			f[3, 4]       & = \frac{5.0 - 8.0}{4 - 3} = -3                                        \\
			f[1, 2, 3]    & = \frac{f[2, 3] - f[1, 2]}{3 - 1} = \frac{4 - 2}{3 - 1} = 1           \\
			f[2, 3, 4]    & = \frac{f[3, 4] - f[2, 3]}{4 - 2} = \frac{(-3) - 4}{2} = -3.5         \\
			f[1, 2, 3, 4] & = \frac{f[2, 3, 4] - f[1, 2, 3]}{4 - 1} = \frac{(-3.5) - 1}{3} = -1.5 \\
		\end{align*}
		差商表为
		\begin{center}
			\begin{tabular}{|c|c|c|c|c|c|}
				\hline
				$i$ & $x_i$ & $f(x_i)$ & $f[x_{i - 1}, x_i]$ & $f[x_{i - 2}, x_{i - 1}, x_i]$ & $f[x_{i - 3}, x_{i - 2}, x_{i - 1}, x_i]$ \\ \hline
				$0$ & $1$   & $2.0$    &                     &                                &                                           \\ \hline
				$1$ & $2$   & $4.0$    & $2$                 &                                &                                           \\ \hline
				$2$ & $3$   & $8.0$    & $4$                 & $1$                            &                                           \\ \hline
				$3$ & $4$   & $5.0$    & $-3$                & $-3.5$                         & $-1.5$                                    \\ \hline
			\end{tabular}
		\end{center}
		相应的Newton插值多项式为
		\begin{align*}
			N_3(x) &= f(1) + (x - 1)f[1, 2] + (x - 1)(x - 2)f[1, 2, 3] + (x - 1)(x - 2)(x - 3)f[1, 2, 3, 4] \\
			&= 2.0 + 2(x - 1) + (x - 1)(x - 2) - 1.5(x - 1)(x - 2)(x -3) \\
			&= -1.5x^3 + 10x^2 -17.5x + 11
		\end{align*}
		计算近似值有
		\begin{align*}
			f(1.5) &\approx N_3(1.5) \\
			&= -1.5 \times 1.5^3 + 10 \times 1.5^2 - 17.5 \times 1.5 + 11 \\
			&= \frac{35}{16} \\
			&= 2.188
		\end{align*}
	}

\end{question}

\begin{question}

	\questiontext{利用数据$f(0) = 2.0$, $f(1) = 0.5$, $f(3) = 0.25$, $f'(3) = 0.6$,构造出三次插值多项式,写出其插值余项,并计算$f(2)$的近似值。}
	\answer{
		定义序列$\{z_0 = 0,\, z_1 = 1,\, z_2 = 3, z_3 = 3\}$,用Newton插值构造Hermite插值多项式,差商表为
		\begin{center}
			\begin{tabular}{|c|c|c|c|c|c|}
				\hline
				$i$ & $z_i$ & $f(z_i)$ & $f[z_{i - 1}, z_i]$ & $f[z_{i - 2}, z_{i - 1}, z_i]$ & $f[z_{i - 3}, z_{i - 2}, z_{i - 1}, z_i]$ \\ \hline
				$0$ & $0$   & $2.0$    &                     &                                &                                           \\ \hline
				$1$ & $1$   & $0.5$    & $-1.5$                 &                                &                                           \\ \hline
				$2$ & $3$   & $0.25$    & $-0.125$                 & $\frac{11}{24}$                            &                                           \\ \hline
				$3$ & $3$   & $0.25$    & $0.6$                & $\frac{29}{80}$                         & $-\frac{23}{720}$                                    \\ \hline
			\end{tabular}
		\end{center}
		其中用$f'(3) = 0.6$代替了$f[z_2, z_3]$即$f[3, 3]$。

		得到三次插值多项式
		\begin{align*}
			H_3(x) &= f(0) + (x - 0)f[0, 1] + (x - 0)(x - 1)f[0, 1, 3] + (x - 0)(x - 1)(x - 3)f[0, 1, 3, 3] \\
			&= 2.0 - 1.5x + \frac{11}{24}x(x - 1) - \frac{23}{720}x(x - 1)(x - 3) \\
			&= \frac{1}{720} (-23x^3 + 422x^2 - 1479x + 1440)
		\end{align*}
		插值余项为
		\begin{align*}
			R_3(x) &= f[x, 0, 1, 3, 3](x - 0)(x - 1)(x - 3)(x - 3) \\
			&= \frac{f^{(4)}(\xi)}{24}x(x - 1)(x - 3)^2, \qquad \xi \in [0, 3]
		\end{align*}
		计算近似值有
		\begin{align*}
			f(2) &\approx H_3(2) \\
			&= \frac{1}{720} (-23 \times 2^3 + 422 \times 2^2 - 1479 \times 2 + 1440) \\
			&= -\frac{7}{360} \\
			&\approx -0.01944
		\end{align*}
	}

\end{question}

\end{document}
