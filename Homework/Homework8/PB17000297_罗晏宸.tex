\documentclass[11pt]{article}
\usepackage[UTF8]{ctex}

%%%%%%%%%%%%%%%%%%%%%%%%%%%%%%%%%%%%%%%%%
% Cleese Assignment
% Structure Specification File
% Version 1.0 (27/5/2018)
%
% This template originates from:
% http://www.LaTeXTemplates.com
%
% Author:
% Vel (vel@LaTeXTemplates.com)
%
% License:
% CC BY-NC-SA 3.0 (http://creativecommons.org/licenses/by-nc-sa/3.0/)
%
%%%%%%%%%%%%%%%%%%%%%%%%%%%%%%%%%%%%%%%%%

%----------------------------------------------------------------------------------------
%	PACKAGES AND OTHER DOCUMENT CONFIGURATIONS
%----------------------------------------------------------------------------------------

\usepackage{lastpage} % Required to determine the last page number for the footer

\usepackage{graphicx} % Required to insert images

\setlength\parindent{0pt} % Removes all indentation from paragraphs

\usepackage[most]{tcolorbox} % Required for boxes that split across pages

\usepackage{booktabs} % Required for better horizontal rules in tables

\usepackage{listings} % Required for insertion of code

\usepackage{etoolbox} % Required for if statements

\usepackage{amsmath}
\usepackage{amsthm}
\usepackage{amssymb}
\usepackage{indentfirst}
\usepackage{diagbox}
\usepackage{subfigure}
\usepackage{float}
\usepackage{xcolor}
\usepackage[colorlinks, linkcolor = black]{hyperref}

\usepackage{enumerate}
\usepackage{enumitem}
\setlist{
    leftmargin = .1\linewidth,
    % rightmargin = .1\linewidth,
    % label=\emph{\alph*}.
}

\setlength{\parindent}{2em}

\usepackage{siunitx}
\sisetup
{
    output-exponent-marker = \ensuremath{\mathrm{E}},
    exponent-product = {},
    retain-explicit-plus,
    retain-zero-exponent,
}
%----------------------------------------------------------------------------------------
%	MARGINS
%----------------------------------------------------------------------------------------

\usepackage{geometry} % Required for adjusting page dimensions and margins

\geometry{
    paper=a4paper, % Change to letterpaper for US letter
    top=3cm, % Top margin
    bottom=3cm, % Bottom margin
    left=2.5cm, % Left margin
    right=2.5cm, % Right margin
    headheight=14pt, % Header height
    footskip=1.4cm, % Space from the bottom margin to the baseline of the footer
    headsep=1.2cm, % Space from the top margin to the baseline of the header
    %showframe, % Uncomment to show how the type block is set on the page
}

%----------------------------------------------------------------------------------------
%	FONT
%----------------------------------------------------------------------------------------

\usepackage[utf8]{inputenc} % Required for inputting international characters
\usepackage[T1]{fontenc} % Output font encoding for international characters

% \usepackage[sfdefault,light]{roboto} % Use the Roboto font

%----------------------------------------------------------------------------------------
%	HEADERS AND FOOTERS
%----------------------------------------------------------------------------------------

\usepackage{fancyhdr} % Required for customising headers and footers

\pagestyle{fancy} % Enable custom headers and footers

\lhead{\small\assignmentClass} % Left header; output the instructor in brackets if one was set
\chead{\small\assignmentTitle} % Centre header
\rhead{\small\ifdef{\assignmentAuthorName}{\assignmentAuthorName}{\ifdef{\assignmentDate}{Due\ \assignmentDate}{}}} % Right header; output the author name if one was set, otherwise the due date if that was set

\lfoot{} % Left footer
\cfoot{\small Page\ \thepage\ of\ \pageref{LastPage}} % Centre footer
\rfoot{} % Right footer

\renewcommand\headrulewidth{0.5pt} % Thickness of the header rule

%----------------------------------------------------------------------------------------
%	MODIFY SECTION STYLES
%----------------------------------------------------------------------------------------

\usepackage{titlesec} % Required for modifying sections

%------------------------------------------------
% Section

\titleformat
{\section} % Section type being modified
[block] % Shape type, can be: hang, block, display, runin, leftmargin, rightmargin, drop, wrap, frame
{\Large\bfseries} % Format of the whole section
{\arabic{section}} % Format of the section label
{6pt} % Space between the title and label
{} % Code before the label

\titlespacing{\section}{0pt}{0.5\baselineskip}{0.5\baselineskip} % Spacing around section titles, the order is: left, before and after

%------------------------------------------------
% Subsection

\titleformat
{\subsection} % Section type being modified
[block] % Shape type, can be: hang, block, display, runin, leftmargin, rightmargin, drop, wrap, frame
{\itshape} % Format of the whole section
{(\arabic{subsection})} % Format of the section label
{4pt} % Space between the title and label
{} % Code before the label

\titlespacing{\subsection}{0pt}{0.5\baselineskip}{0.5\baselineskip} % Spacing around section titles, the order is: left, before and after

\renewcommand\thesubsection{(\arabic{subsection})}

%----------------------------------------------------------------------------------------
%	CUSTOM QUESTION COMMANDS/ENVIRONMENTS
%----------------------------------------------------------------------------------------



% Command to print an assignment section title to split an assignment into major parts
\newcommand{\assignmentSection}[1]{
    \newpage
    {
        \centering % Centre the section title
        \vspace{2\baselineskip} % Whitespace before the entire section title

        \rule{0.8\textwidth}{0.5pt} % Horizontal rule

        \vspace{0.75\baselineskip} % Whitespace before the section title
        {\LARGE \textsc{#1}} % Section title, forced to be uppercase

        \rule{0.8\textwidth}{0.5pt} % Horizontal rule

        \vspace{\baselineskip} % Whitespace after the entire section title
    }
    \setcounter{section}{0}

}

%----------------------------------------------------------------------------------------
%	TITLE PAGE
%----------------------------------------------------------------------------------------

\author{\textbf{\assignmentNo\ \assignmentAuthorName}} % Set the default title page author field
\date{} % Don't use the default title page date field

\title{
    \thispagestyle{empty} % Suppress headers and footers
    \vspace{0.2\textheight} % Whitespace before the title
    \textbf{\assignmentClass}\\[5pt]
    \texttt{\assignmentTitle}\\[-4pt]
    % \ifdef{\assignmentSubTitle}{\texttt{\assignmentSubTitle}}{}
    \ifdef{\assignmentDate}{\assignmentDate}{} % If a due date is supplied, output it
    \ifdef{\assignmentClassInstructor}{{\large \textit{\assignmentClassInstructor}}}{} % If an instructor is supplied, output it
    \vspace{0.32\textheight} % Whitespace before the author name
}


\newcommand{\assignmentQuestionName}{Question}
\newcommand{\assignmentClass}{计算方法B}
\newcommand{\assignmentTitle}{Homework\ \#8}
\newcommand{\assignmentDate}{2020.5.15}
\newcommand{\assignmentNo}{PB17000297}
\newcommand{\assignmentAuthorName}{罗晏宸}

\begin{document}

\maketitle

\thispagestyle{empty}

\newpage

\begin{question}

    \questiontext
    {
    给定函数$f(x)$离散值如下:
    \begin{table}[h]
        \centering
        \begin{tabular}{|c|c|c|c|c|c|}
            \hline
            $x$    & $1.0$ & $1.2$ & $1.4$ & $1.6$ & $1.8$ \\ \hline
            $f(x)$ & $3.2$ & $3.5$ & $5.0$ & $5.2$ & $4.8$ \\ \hline
        \end{tabular}
    \end{table}
    分别用复化梯形和复化 Simpson 公式计算$\displaystyle \int_{1.0}^{1.8}\!{f(x)\,\text{d}x}$。
    }
    \answer
    {
        题设给定了 5 个数据点,将区间$[1.0, 1.8]$等距分割为 4 等分,设$a = 1.0,\ b = 1.8, h = 0.2$,则
        $$
            x_i = a + ih,\quad i = 0,\, 1,\, \cdots,\, 4
        $$
        复化梯形积分为
        \begin{align*}
            T(h) & = T(0.2) = T_4(f)                                                                              \\
                 & = h \left[\frac{1}{2}f(a) + \sum_{i = 1}^{3}{f(a + ih)} + \frac{1}{2}f(b)\right]               \\
                 & = 0.2 \times \left[\frac{1}{2} \times 3.2 + (3.5 + 5.0 + 5.2) + \frac{1}{2} \times 4.8 \right] \\
                 & = 3.54
        \end{align*}
        复化 Simpson 积分为
        \begin{align*}
            S_4(f)
             & = \frac{h}{3} \left[f(a) + 4 \sum_{i = 0}^{1}{f(x_{2i + 1})} + 2\sum_{i = 1}^{1}{f(x_{2i})} + f(b)\right] \\
             & = \frac{0.2}{3} \times \left[3.2 + 4 \times (3.5 + 5.2) + 2 \times 5.0 + 4.8\right]                       \\
             & = 3.52
        \end{align*}
    }
\end{question}

\begin{question}
    \questiontext
    {
    构造积分$\bar{I}(f) = \displaystyle \int_{-h}^{2h}\!{f(x)\,\text{d}x}$的数值积分公式
    \begin{equation*}
        I(f) = a_{-1}f(-h) + a_0f(0) + a_1f(2h) \quad (h > 0)
    \end{equation*}
    使其具有尽可能高的代数精度,该公式的代数精度是多少?
    }
    \answer
    {
    取$f(x) = x^k$时,令
    \begin{align*}
        \bar{I}(f) & = \int_{-h}^{2h}\!{x^0\,\text{d}x} = \left(2h - (-h)\right) = a_{-1} + a_0 + a_1 = I(f)      \\
        \bar{I}(f) & = \int_{-h}^{2h}\!{x^k\,\text{d}x} = \frac{1}{k + 1}\left((2h)^{k + 1} - (-h)^{k + 1}\right) \\
                   & = a_{-1}(-h)^k + a_1(2h)^k = I(f),\quad k = 1,\cdots, m
    \end{align*}
    这是关于$a_{-1},\, a_0,\, a_1$的线性方程组,理论上由$m = 2$时的 3 个方程即可解得确定值
    \begin{align*}
                    & \left\{
        \begin{aligned}
            a_{-1} + a_0 + a_1 & = 3h           \\
            -a_{-1} + 2a_1     & = \frac{3}{2}h \\
            a_{-1} + 4a_1      & = 3h
        \end{aligned}
        \right.               \\
        \Rightarrow & \left\{
        \begin{aligned}
            a_{-1} & = 0            \\
            a_0    & = \frac{9}{4}h \\
            a_1    & =\frac{3}{4}h
        \end{aligned}
        \right.
    \end{align*}
    代入$k = 3$时的方程
    \begin{equation*}
        \bar{I}(f) = \int_{-h}^{2h}\!{x^3\,\text{d}x} = \frac{1}{4}\left((2h)^{4} - (-h)^{4}\right) = \frac{15}{4}h^4 \neq 12h^4 = a_{-1}(-h)^3 + a_1(2h)^3 = I(f)
    \end{equation*}
    因此积分$\bar{I}(f) = \displaystyle \int_{-h}^{2h}\!{f(x)\,\text{d}x}$的数值积分公式
    \begin{equation*}
        I(f) = h\left[\frac{9}{4}f(0) + \frac{3}{4}f(2h)\right] \quad (h > 0)
    \end{equation*}
    具有最高的 2 阶代数精度。
    }
\end{question}

\begin{question}
    \questiontext
    {
    记 $I(f) = \displaystyle \int_{-2}^{2}\!{f(x)\,\text{d}x}$,设$S\left(f(x)\right)$为其数值积分公式,其中,$I(f) \approx S\left(f(x)\right) = Af(-\alpha) + Bf(0) + Cf(\alpha)$。
    }
    \begin{subquestion}{试确定参数$A,\ B,\ C,\ \alpha$,使得该数值积分公式具有尽可能高的代数精度,并求该公式的代数精度(需给出求解过程)}
        \answer
        {
        取$f(x) = x^k$时,令
        \begin{align*}
            I(f) & = \int_{-2}^{2}\!{x^0\,\text{d}x} = \left(2 - (-2)\right) = A + B + C = S\left(f(x)\right) \\
            I(f) & = \int_{-2}^{2}\!{x^k\,\text{d}x} = \frac{1}{k + 1}\left(2^{k + 1} - (-2)^{k + 1}\right)   \\
                 & = A(-\alpha)^k + C\alpha^k = S\left(f(x)\right),\quad k = 1,\cdots, m
        \end{align*}
        这是关于$A,\ B,\ C,\ \alpha$的非线性方程组,当$m = 4$时方程组由如下 5 个方程组成
        \begin{align*}
                        & \left\{
            \begin{aligned}
                A + B + C                & = 4            \\
                -\alpha A + \alpha C     & = 0            \\
                \alpha^2 A + \alpha^2 C  & = \frac{16}{3} \\
                -\alpha^3 A + \alpha^3 C & = 0            \\
                \alpha^4 A + \alpha^4 C  & = \frac{64}{5}
            \end{aligned}
            \right.               \\
            \Rightarrow & \left\{
            \begin{aligned}
                A        & = \frac{10}{9} \\
                B        & = \frac{16}{9} \\
                C        & = \frac{10}{9} \\
                \alpha^2 & = \frac{12}{5}
            \end{aligned}
            \right.
        \end{align*}
        由数值积分公式形式的对称性,不妨设$\alpha > 0$,即$\alpha = \dfrac{2\sqrt{15}}{5}$。代入$k = 5$及$k = 6$时的方程
        \begin{align*}
            I(f) = \int_{-2}^{2}\!{x^5\,\text{d}x} = 0             & =0= \alpha^6(-A + C) = S\left(f(x)\right)                             \\
            I(f) = \int_{-2}^{2}\!{x^6\,\text{d}x} = \frac{256}{7} & \neq \frac{1728}{125} = A(-\alpha)^6 + C\alpha^6 = S\left(f(x)\right)
        \end{align*}
        因此$I(f) = \displaystyle \int_{-2}^{2}\!{f(x)\,\text{d}x}$的数值积分公式
        \begin{equation*}
            S\left(f(x)\right) = \frac{10}{9}f\left(-\frac{2\sqrt{15}}{5}\right) + \frac{16}{9}f(0) + \frac{10}{9}f\left(\frac{2\sqrt{15}}{5}\right)
        \end{equation*}
        具有最高的 5 阶代数精度。
        }
    \end{subquestion}
    \begin{subquestion}{设$f(x)$足够光滑(可微),求该数值积分公式的误差}
        \answer
        {
            对于给定的被积函数$f(x)$在$[-2, 2]$上的点列$\left\{-\dfrac{2\sqrt{15}}{5},\, 0,\, \dfrac{2\sqrt{15}}{5}\right\}$,作 Lagrange 插值多项式 $L_2(x)$,由数值积分公式有 5 阶代数精度,其误差就是对插值误差函数的积分
            \begin{align*}
                E_2(f) = \int_{-2}^{2}\!{R_2(x)\,\text{d}x} = \frac{1}{6}\int_{-2}^{2}\!{f'''(\xi)\left(x + \frac{2\sqrt{15}}{5}\right)x\left(x - \frac{2\sqrt{15}}{5}\right)\,\text{d}x}
            \end{align*}
            注意到这里$\xi = \xi(x) \in [-2, 2]$是$x$的函数,由积分第一中值定理
            \begin{align*}
                E_2(f) & = \frac{1}{6}\left(\int_{-2}^{-\frac{2\sqrt{15}}{5}}\! + \int_{-\frac{2\sqrt{15}}{5}}^{0}\! + \int_{0}^{\frac{2\sqrt{15}}{5}}\! + \int_{\frac{2\sqrt{15}}{5}}^{2}\!\right){\left[f'''(\xi)\left(x + \frac{2\sqrt{15}}{5}\right)x\left(x - \frac{2\sqrt{15}}{5}\right)\,\text{d}x\right]} \\
                       & = \frac{f'''(\eta_1)}{6}\int_{-2}^{-\frac{2\sqrt{15}}{5}}\!{\left[\left(x + \frac{2\sqrt{15}}{5}\right)x\left(x - \frac{2\sqrt{15}}{5}\right)\,\text{d}x\right]}                                                                                                                         \\
                       & \phantom{1+1+1+1} +\frac{\eta_2f'''(\eta_2)}{6}\int_{-\frac{2\sqrt{15}}{5}}^{\frac{2\sqrt{15}}{5}}\!{\left[\left(x + \frac{2\sqrt{15}}{5}\right)\left(x - \frac{2\sqrt{15}}{5}\right)\,\text{d}x\right]}                                                                                 \\
                       & \phantom{1+1+1+1} + \frac{f'''(\eta_3)}{6}\int_{\frac{2\sqrt{15}}{5}}^{2}\!{\left[\left(x + \frac{2\sqrt{15}}{5}\right)x\left(x - \frac{2\sqrt{15}}{5}\right)\,\text{d}x\right]}                                                                                                         \\
                       & = \frac{f'''(\eta_1)}{6} \times \left(- \frac{16}{25}\right) + \frac{\eta_2f'''(\eta_2)}{6} \times \left(-\frac{32\sqrt{15}}{25}\right) + \frac{f'''(\eta_3)}{6} \times \frac{16}{25}                                                                                                    \\
                       & = \frac{8}{75}\left(- f'''(\eta_1) - 2\sqrt{15}\eta_2f'''(\eta_2) + f'''(\eta_3) \right),\quad \eta_1,\, \eta_2,\, \eta_3 \in [-2, 2]
            \end{align*}
        }
    \end{subquestion}
\end{question}
\end{document}
