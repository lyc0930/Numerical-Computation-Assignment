\documentclass[11pt]{article}
\usepackage[UTF8]{ctex}

%%%%%%%%%%%%%%%%%%%%%%%%%%%%%%%%%%%%%%%%%
% Cleese Assignment
% Structure Specification File
% Version 1.0 (27/5/2018)
%
% This template originates from:
% http://www.LaTeXTemplates.com
%
% Author:
% Vel (vel@LaTeXTemplates.com)
%
% License:
% CC BY-NC-SA 3.0 (http://creativecommons.org/licenses/by-nc-sa/3.0/)
%
%%%%%%%%%%%%%%%%%%%%%%%%%%%%%%%%%%%%%%%%%

%----------------------------------------------------------------------------------------
%	PACKAGES AND OTHER DOCUMENT CONFIGURATIONS
%----------------------------------------------------------------------------------------

\usepackage{lastpage} % Required to determine the last page number for the footer

\usepackage{graphicx} % Required to insert images

\setlength\parindent{0pt} % Removes all indentation from paragraphs

\usepackage[most]{tcolorbox} % Required for boxes that split across pages

\usepackage{booktabs} % Required for better horizontal rules in tables

\usepackage{listings} % Required for insertion of code

\usepackage{etoolbox} % Required for if statements

\usepackage{amsmath}
\usepackage{amsthm}
\usepackage{amssymb}
\usepackage{indentfirst}
\usepackage{diagbox}
\usepackage{subfigure}
\usepackage{float}
\usepackage{xcolor}
\usepackage[colorlinks, linkcolor = black]{hyperref}

\usepackage{enumerate}
\usepackage{enumitem}
\setlist{
    leftmargin = .1\linewidth,
    % rightmargin = .1\linewidth,
    % label=\emph{\alph*}.
}

\setlength{\parindent}{2em}

\usepackage{siunitx}
\sisetup
{
    output-exponent-marker = \ensuremath{\mathrm{E}},
    exponent-product = {},
    retain-explicit-plus,
    retain-zero-exponent,
}
%----------------------------------------------------------------------------------------
%	MARGINS
%----------------------------------------------------------------------------------------

\usepackage{geometry} % Required for adjusting page dimensions and margins

\geometry{
    paper=a4paper, % Change to letterpaper for US letter
    top=3cm, % Top margin
    bottom=3cm, % Bottom margin
    left=2.5cm, % Left margin
    right=2.5cm, % Right margin
    headheight=14pt, % Header height
    footskip=1.4cm, % Space from the bottom margin to the baseline of the footer
    headsep=1.2cm, % Space from the top margin to the baseline of the header
    %showframe, % Uncomment to show how the type block is set on the page
}

%----------------------------------------------------------------------------------------
%	FONT
%----------------------------------------------------------------------------------------

\usepackage[utf8]{inputenc} % Required for inputting international characters
\usepackage[T1]{fontenc} % Output font encoding for international characters

% \usepackage[sfdefault,light]{roboto} % Use the Roboto font

%----------------------------------------------------------------------------------------
%	HEADERS AND FOOTERS
%----------------------------------------------------------------------------------------

\usepackage{fancyhdr} % Required for customising headers and footers

\pagestyle{fancy} % Enable custom headers and footers

\lhead{\small\assignmentClass} % Left header; output the instructor in brackets if one was set
\chead{\small\assignmentTitle} % Centre header
\rhead{\small\ifdef{\assignmentAuthorName}{\assignmentAuthorName}{\ifdef{\assignmentDate}{Due\ \assignmentDate}{}}} % Right header; output the author name if one was set, otherwise the due date if that was set

\lfoot{} % Left footer
\cfoot{\small Page\ \thepage\ of\ \pageref{LastPage}} % Centre footer
\rfoot{} % Right footer

\renewcommand\headrulewidth{0.5pt} % Thickness of the header rule

%----------------------------------------------------------------------------------------
%	MODIFY SECTION STYLES
%----------------------------------------------------------------------------------------

\usepackage{titlesec} % Required for modifying sections

%------------------------------------------------
% Section

\titleformat
{\section} % Section type being modified
[block] % Shape type, can be: hang, block, display, runin, leftmargin, rightmargin, drop, wrap, frame
{\Large\bfseries} % Format of the whole section
{\arabic{section}} % Format of the section label
{6pt} % Space between the title and label
{} % Code before the label

\titlespacing{\section}{0pt}{0.5\baselineskip}{0.5\baselineskip} % Spacing around section titles, the order is: left, before and after

%------------------------------------------------
% Subsection

\titleformat
{\subsection} % Section type being modified
[block] % Shape type, can be: hang, block, display, runin, leftmargin, rightmargin, drop, wrap, frame
{\itshape} % Format of the whole section
{(\arabic{subsection})} % Format of the section label
{4pt} % Space between the title and label
{} % Code before the label

\titlespacing{\subsection}{0pt}{0.5\baselineskip}{0.5\baselineskip} % Spacing around section titles, the order is: left, before and after

\renewcommand\thesubsection{(\arabic{subsection})}

%----------------------------------------------------------------------------------------
%	CUSTOM QUESTION COMMANDS/ENVIRONMENTS
%----------------------------------------------------------------------------------------



% Command to print an assignment section title to split an assignment into major parts
\newcommand{\assignmentSection}[1]{
    \newpage
    {
        \centering % Centre the section title
        \vspace{2\baselineskip} % Whitespace before the entire section title

        \rule{0.8\textwidth}{0.5pt} % Horizontal rule

        \vspace{0.75\baselineskip} % Whitespace before the section title
        {\LARGE \textsc{#1}} % Section title, forced to be uppercase

        \rule{0.8\textwidth}{0.5pt} % Horizontal rule

        \vspace{\baselineskip} % Whitespace after the entire section title
    }
    \setcounter{section}{0}

}

%----------------------------------------------------------------------------------------
%	TITLE PAGE
%----------------------------------------------------------------------------------------

\author{\textbf{\assignmentNo\ \assignmentAuthorName}} % Set the default title page author field
\date{} % Don't use the default title page date field

\title{
    \thispagestyle{empty} % Suppress headers and footers
    \vspace{0.2\textheight} % Whitespace before the title
    \textbf{\assignmentClass}\\[5pt]
    \texttt{\assignmentTitle}\\[-4pt]
    % \ifdef{\assignmentSubTitle}{\texttt{\assignmentSubTitle}}{}
    \ifdef{\assignmentDate}{\assignmentDate}{} % If a due date is supplied, output it
    \ifdef{\assignmentClassInstructor}{{\large \textit{\assignmentClassInstructor}}}{} % If an instructor is supplied, output it
    \vspace{0.32\textheight} % Whitespace before the author name
}


\newcommand{\assignmentQuestionName}{Question}
\newcommand{\assignmentClass}{计算方法B}
\newcommand{\assignmentTitle}{Homework\ \#7}
\newcommand{\assignmentDate}{2020.5.6}
\newcommand{\assignmentNo}{PB17000297}
\newcommand{\assignmentAuthorName}{罗晏宸}

\begin{document}

\maketitle

\thispagestyle{empty}

\newpage

\begin{question}

    \questiontext
    {
        设有线性代数方程组 $Ax = b$,其中,
        \begin{equation*}
            A =
            \begin{pmatrix}
                \phantom{-}2 & -1           & \phantom{-}0 & \phantom{-}0 \\
                -1           & \phantom{-}2 & -1           & \phantom{-}0 \\
                \phantom{-}0 & -1           & \phantom{-}2 & -1           \\
                \phantom{-}0 & \phantom{-}0 & -1           & \phantom{-}2
            \end{pmatrix}
            ,\quad
            b =
            \begin{pmatrix}
                2 \\
                2 \\
                2 \\
                2
            \end{pmatrix}
        \end{equation*}
    }
    \begin{subquestion}{求 Jacobi 迭代的迭代矩阵及相应的迭代格式;}
        \answer
        {
            令
            \begin{equation*}
                D = \operatorname{diag}{\left\{a_{11}, a_{22}, a_{33}, a_{44}\right\}} =\operatorname{diag}{\left\{2, 2, 2, 2\right\}}
            \end{equation*}
            $D$可逆,由$Ax = (D + A - D)x = b$得到等价方程组$Dx = (D - A)x + b$,迭代格式的矩阵形式为
            \begin{equation*}
                x^{(k + 1)} = Rx^{(k)} + g
            \end{equation*}
            其中$R$是迭代矩阵
            \begin{equation*}
                R = \mathbf{I} - D^{-1}A =
                \begin{pmatrix}[1.7]
                    0            & \dfrac{1}{2} & 0            & 0            \\
                    \dfrac{1}{2} & 0            & \dfrac{1}{2} & 0            \\
                    0            & \dfrac{1}{2} & 0            & \dfrac{1}{2} \\
                    0            & 0            & \dfrac{1}{2} & 0
                \end{pmatrix}
                ,\quad
                g = D^{-1}b =
                \begin{pmatrix}[1.7]
                    1 \\
                    1 \\
                    1 \\
                    1
                \end{pmatrix}
            \end{equation*}
            相应的迭代格式为
            \begin{equation*}
                \left\{
                \begin{aligned}
                    x_1^{(k + 1)} & = 0.5x_2^{(k)} + 1                \\
                    x_2^{(k + 1)} & = 0.5x_1^{(k)} + 0.5x_3^{(k)} + 1 \\
                    x_3^{(k + 1)} & = 0.5x_2^{(k)} + 0.5x_4^{(k)} + 1 \\
                    x_4^{(k + 1)} & = 0.5x_3^{(k)} + 1
                \end{aligned}
                \right.
            \end{equation*}
        }
    \end{subquestion}

    \begin{subquestion}{讨论此时 Jacobi 迭代(方法)的收敛性。}
        \answer
        {
            题设系数矩阵$A$满足
            \begin{align*}
                \left|a_{11}\right| = 2 > \sum_{j \neq 1}{\left|a_{1j}\right|} = 1 \\
                \left|a_{22}\right| = 2 = \sum_{j \neq 2}{\left|a_{2j}\right|} = 2 \\
                \left|a_{33}\right| = 2 = \sum_{j \neq 3}{\left|a_{3j}\right|} = 2 \\
                \left|a_{44}\right| = 2 > \sum_{j \neq 4}{\left|a_{4j}\right|} = 1
            \end{align*}
            可知$A$是行对角优但不是严格行对角优的。同时注意到$A^\mathbf{T} = A$,即$A$是对称矩阵,可知$A$是列对角优但不是严格列对角优的,因此$A$不是严格对角优的,无法直接判断 Jacobi 迭代的收敛性。

            下面计算迭代矩阵$R$的谱半径$\rho(R)$,设$\lambda$为$R$的特征值,则有
            \begin{align*}
                            &  & \det{(R - \lambda \mathbf{I})}               & = 0 &  &  & \\
                \Rightarrow &  & \begin{vmatrix}[1.7]
                    -\lambda     & \dfrac{1}{2} & 0            & 0            \\
                    \dfrac{1}{2} & -\lambda     & \dfrac{1}{2} & 0            \\
                    0            & \dfrac{1}{2} & -\lambda     & \dfrac{1}{2} \\
                    0            & 0            & \dfrac{1}{2} & -\lambda
                \end{vmatrix}                    & = 0 &  &  & \\
                \Rightarrow &  & \lambda ^4-\frac{3}{4}\lambda^2+\frac{1}{16} & = 0 &  &  &
            \end{align*}
            解得
            \begin{equation*}
                \lambda_1 = - \frac{1 + \sqrt{5}}{4},\, \lambda_2 = \frac{1 - \sqrt{5}}{4},\, \lambda_3 = - \frac{1 - \sqrt{5}}{4},\, \lambda_4 = \frac{1 + \sqrt{5}}{4}
            \end{equation*}
            进而有$R$的谱半径
            \begin{align*}
                \rho(R) & = \max_i{\left|\lambda_i\right|}                                                                                           \\
                        & = \max{\left\{\frac{1 + \sqrt{5}}{4},\, \frac{\sqrt{5} - 1}{4},\, \frac{\sqrt{5} - 1}{4},\,\frac{1 + \sqrt{5}}{4}\right\}} \\
                        & = \frac{1 + \sqrt{5}}{4} < 1
            \end{align*}
            因此迭代是收敛的。
        }
    \end{subquestion}

\end{question}

\begin{question}
    \questiontext
    {
        设有线性代数方程组
        \begin{equation*}
            \left\{
            \begin{aligned}
                \phantom{-}5x_1 - 3x_2 + 2x_3 & = 5 \\
                -3x_1 + 5x_2 + 2x_3           & = 5 \\
                \phantom{-}2x_1 + 2x_2 + 7x_3 & = 7
            \end{aligned}
            \right.
        \end{equation*}
    }
    \begin{subquestion}{写出 Gauss-Seidel 迭代的分量形式;}
        \answer
        {
            将第$i$行的变量$x_i$留在方程左边,并除以系数$a_{ii}$,系数矩阵的下三角元素冠以$k+1$得到迭代格式的分量形式为
            \begin{equation*}
                \left\{
                \begin{aligned}
                    x_1^{(k + 1)} & = \phantom{-}\dfrac{3}{5}x_2^{(k)\phantom{+1}} - \dfrac{2}{5}x_3^{(k)\phantom{+1}} + 1 \\
                    x_2^{(k + 1)} & = \phantom{-}\dfrac{3}{5}x_1^{(k + 1)} - \dfrac{2}{5}x_2^{(k)\phantom{+1}} + 1         \\
                    x_3^{(k + 1)} & = -\dfrac{2}{7}x_1^{(k + 1)} - \dfrac{2}{7}x_2^{(k + 1)} + 1
                \end{aligned}
                \right.
            \end{equation*}
        }
    \end{subquestion}
    \begin{subquestion}{求 Gauss-Seidel 迭代的分裂矩阵(splitting matrix)及迭代矩阵(iteration matrix);}
        \answer
        {
            设原方程组的矩阵形式$Ax = b$,其中
            \begin{equation*}
                A =
                \begin{pmatrix}
                    \phantom{-}5 & -3           & \phantom{-}2 \\
                    -3           & \phantom{-}5 & \phantom{-}2 \\
                    \phantom{-}2 & \phantom{-}2 & \phantom{-}7
                \end{pmatrix}
                ,\quad
                b =
                \begin{pmatrix}
                    5 \\
                    5 \\
                    7
                \end{pmatrix}
            \end{equation*}
            分解系数矩阵
            \begin{align*}
                A & = D + L + U \\
                  & =
                \begin{pmatrix}
                    5 & 0 & 0 \\
                    0 & 5 & 0 \\
                    0 & 0 & 7
                \end{pmatrix}
                +
                \begin{pmatrix}
                    \phantom{-}0 & \phantom{-}0 & \phantom{-}0 \\
                    -3           & \phantom{-}0 & \phantom{-}0 \\
                    \phantom{-}2 & \phantom{-}2 & \phantom{-}0
                \end{pmatrix}
                +
                \begin{pmatrix}
                    \phantom{-}0 & -3           & \phantom{-}2 \\
                    \phantom{-}0 & \phantom{-}0 & \phantom{-}2 \\
                    \phantom{-}0 & \phantom{-}0 & \phantom{-}0
                \end{pmatrix}
            \end{align*}
            写成等价矩阵表达式
            \begin{equation*}
                (D + L)x = -Ux + b
            \end{equation*}
            记分裂矩阵
            \begin{equation*}
                Q = D + L =
                \begin{pmatrix}
                    \phantom{-}5 & \phantom{-}0 & \phantom{-}0 \\
                    -3           & \phantom{-}5 & \phantom{-}0 \\
                    \phantom{-}2 & \phantom{-}2 & \phantom{-}7
                \end{pmatrix}
            \end{equation*}
            构造迭代形式
            \begin{equation*}
                Qx^{(k + 1)} = -Ux^{(k)} + b
            \end{equation*}
            记 Gauss-Seidel 迭代的矩阵形式为

            \begin{equation*}
                x^{(k + 1)} = Sx^{(k)} + f
            \end{equation*}
            其中$S$是迭代矩阵
            \begin{equation*}
                S = -(D + L)^{-1}U = \mathbf{I} - Q^{-1}A =
                \begin{pmatrix}[1.7]
                    0 & \phantom{-}\dfrac{3}{5}  & -\dfrac{2}{5}              \\
                    0 & \phantom{-}\dfrac{9}{25} & -\dfrac{16}{25}            \\
                    0 & -\dfrac{48}{175}         & \phantom{-}\dfrac{52}{175}
                \end{pmatrix}
                ,\quad f = Q^{-1}b =
                \begin{pmatrix}[1.7]
                    1            \\
                    \dfrac{8}{5} \\
                    \dfrac{9}{35}
                \end{pmatrix}
            \end{equation*}
        }
    \end{subquestion}
    \begin{subquestion}{讨论 Gauss-Seidel 迭代的收敛性(请给出理由或证明)。}
        \answer
        {
            观察原方程的系数矩阵
            \begin{equation*}
                A =
                \begin{pmatrix}
                    \phantom{-}5 & -3           & \phantom{-}2 \\
                    -3           & \phantom{-}5 & \phantom{-}2 \\
                    \phantom{-}2 & \phantom{-}2 & \phantom{-}7
                \end{pmatrix}
            \end{equation*}
            其满足
            \begin{align*}
                \left|a_{11}\right| = 5 = \sum_{j \neq 1}{\left|a_{1j}\right|} = 5 \\
                \left|a_{22}\right| = 5 = \sum_{j \neq 2}{\left|a_{2j}\right|} = 5 \\
                \left|a_{33}\right| = 7 > \sum_{j \neq 3}{\left|a_{3j}\right|} = 4
            \end{align*}
            可知$A$是行对角优但不是严格行对角优的。同时注意到$A^\mathbf{T} = A$,即$A$是对称矩阵,可知$A$是列对角优但不是严格列对角优的,因此$A$不是严格对角优的,无法说明迭代的收敛性。

            下面证明$A$是正定的,$A$的各阶顺序主子式为
            \begin{align*}
                [A]_{\{1\},\{1\}}         & =
                \begin{vmatrix}
                    5
                \end{vmatrix}
                = 5                                                                            \\
                [A]_{\{1,2\},\{1,2\}}     & =
                \begin{vmatrix}
                    \phantom{-}5 & -3           \\
                    -3           & \phantom{-}5
                \end{vmatrix}
                = 5 \times 5 - (-3) \times (-3)  = 16                                          \\
                [A]_{\{1,2,3\},\{1,2,3\}} & =
                \begin{vmatrix}
                    \phantom{-}5 & -3           & \phantom{-}2 \\
                    -3           & \phantom{-}5 & \phantom{-}2 \\
                    \phantom{-}2 & \phantom{-}2 & \phantom{-}7
                \end{vmatrix}                                                     \\
                                          & = 5 \times
                \begin{vmatrix}
                    \phantom{-}5 & \phantom{-}2 \\
                    \phantom{-}2 & \phantom{-}7
                \end{vmatrix}
                - (-3) \times
                \begin{vmatrix}
                    -3           & \phantom{-}2 \\
                    \phantom{-}2 & \phantom{-}7
                \end{vmatrix}
                + 2 \times
                \begin{vmatrix}
                    -3           & \phantom{-}5 \\
                    \phantom{-}2 & \phantom{-}2
                \end{vmatrix}                                                     \\
                                          & = 5 \times 31 - (-3) \times (-25) + 2 \times (-16) \\
                                          & = 48
            \end{align*}
            $A$的各阶顺序主子式均为正,因此$A$是正定的,所以 Gauss-Seidel 迭代收敛。
        }
    \end{subquestion}

\end{question}

\end{document}
