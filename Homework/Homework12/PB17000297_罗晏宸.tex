\documentclass[11pt]{article}
\usepackage[UTF8]{ctex}

%%%%%%%%%%%%%%%%%%%%%%%%%%%%%%%%%%%%%%%%%
% Cleese Assignment
% Structure Specification File
% Version 1.0 (27/5/2018)
%
% This template originates from:
% http://www.LaTeXTemplates.com
%
% Author:
% Vel (vel@LaTeXTemplates.com)
%
% License:
% CC BY-NC-SA 3.0 (http://creativecommons.org/licenses/by-nc-sa/3.0/)
%
%%%%%%%%%%%%%%%%%%%%%%%%%%%%%%%%%%%%%%%%%

%----------------------------------------------------------------------------------------
%	PACKAGES AND OTHER DOCUMENT CONFIGURATIONS
%----------------------------------------------------------------------------------------

\usepackage{lastpage} % Required to determine the last page number for the footer

\usepackage{graphicx} % Required to insert images

\setlength\parindent{0pt} % Removes all indentation from paragraphs

\usepackage[most]{tcolorbox} % Required for boxes that split across pages

\usepackage{booktabs} % Required for better horizontal rules in tables

\usepackage{listings} % Required for insertion of code

\usepackage{etoolbox} % Required for if statements

\usepackage{amsmath}
\usepackage{amsthm}
\usepackage{amssymb}
\usepackage{indentfirst}
\usepackage{diagbox}
\usepackage{subfigure}
\usepackage{float}
\usepackage{xcolor}
\usepackage[colorlinks, linkcolor = black]{hyperref}

\usepackage{enumerate}
\usepackage{enumitem}
\setlist{
    leftmargin = .1\linewidth,
    % rightmargin = .1\linewidth,
    % label=\emph{\alph*}.
}

\setlength{\parindent}{2em}

\usepackage{siunitx}
\sisetup
{
    output-exponent-marker = \ensuremath{\mathrm{E}},
    exponent-product = {},
    retain-explicit-plus,
    retain-zero-exponent,
}
%----------------------------------------------------------------------------------------
%	MARGINS
%----------------------------------------------------------------------------------------

\usepackage{geometry} % Required for adjusting page dimensions and margins

\geometry{
    paper=a4paper, % Change to letterpaper for US letter
    top=3cm, % Top margin
    bottom=3cm, % Bottom margin
    left=2.5cm, % Left margin
    right=2.5cm, % Right margin
    headheight=14pt, % Header height
    footskip=1.4cm, % Space from the bottom margin to the baseline of the footer
    headsep=1.2cm, % Space from the top margin to the baseline of the header
    %showframe, % Uncomment to show how the type block is set on the page
}

%----------------------------------------------------------------------------------------
%	FONT
%----------------------------------------------------------------------------------------

\usepackage[utf8]{inputenc} % Required for inputting international characters
\usepackage[T1]{fontenc} % Output font encoding for international characters

% \usepackage[sfdefault,light]{roboto} % Use the Roboto font

%----------------------------------------------------------------------------------------
%	HEADERS AND FOOTERS
%----------------------------------------------------------------------------------------

\usepackage{fancyhdr} % Required for customising headers and footers

\pagestyle{fancy} % Enable custom headers and footers

\lhead{\small\assignmentClass} % Left header; output the instructor in brackets if one was set
\chead{\small\assignmentTitle} % Centre header
\rhead{\small\ifdef{\assignmentAuthorName}{\assignmentAuthorName}{\ifdef{\assignmentDate}{Due\ \assignmentDate}{}}} % Right header; output the author name if one was set, otherwise the due date if that was set

\lfoot{} % Left footer
\cfoot{\small Page\ \thepage\ of\ \pageref{LastPage}} % Centre footer
\rfoot{} % Right footer

\renewcommand\headrulewidth{0.5pt} % Thickness of the header rule

%----------------------------------------------------------------------------------------
%	MODIFY SECTION STYLES
%----------------------------------------------------------------------------------------

\usepackage{titlesec} % Required for modifying sections

%------------------------------------------------
% Section

\titleformat
{\section} % Section type being modified
[block] % Shape type, can be: hang, block, display, runin, leftmargin, rightmargin, drop, wrap, frame
{\Large\bfseries} % Format of the whole section
{\arabic{section}} % Format of the section label
{6pt} % Space between the title and label
{} % Code before the label

\titlespacing{\section}{0pt}{0.5\baselineskip}{0.5\baselineskip} % Spacing around section titles, the order is: left, before and after

%------------------------------------------------
% Subsection

\titleformat
{\subsection} % Section type being modified
[block] % Shape type, can be: hang, block, display, runin, leftmargin, rightmargin, drop, wrap, frame
{\itshape} % Format of the whole section
{(\arabic{subsection})} % Format of the section label
{4pt} % Space between the title and label
{} % Code before the label

\titlespacing{\subsection}{0pt}{0.5\baselineskip}{0.5\baselineskip} % Spacing around section titles, the order is: left, before and after

\renewcommand\thesubsection{(\arabic{subsection})}

%----------------------------------------------------------------------------------------
%	CUSTOM QUESTION COMMANDS/ENVIRONMENTS
%----------------------------------------------------------------------------------------



% Command to print an assignment section title to split an assignment into major parts
\newcommand{\assignmentSection}[1]{
    \newpage
    {
        \centering % Centre the section title
        \vspace{2\baselineskip} % Whitespace before the entire section title

        \rule{0.8\textwidth}{0.5pt} % Horizontal rule

        \vspace{0.75\baselineskip} % Whitespace before the section title
        {\LARGE \textsc{#1}} % Section title, forced to be uppercase

        \rule{0.8\textwidth}{0.5pt} % Horizontal rule

        \vspace{\baselineskip} % Whitespace after the entire section title
    }
    \setcounter{section}{0}

}

%----------------------------------------------------------------------------------------
%	TITLE PAGE
%----------------------------------------------------------------------------------------

\author{\textbf{\assignmentNo\ \assignmentAuthorName}} % Set the default title page author field
\date{} % Don't use the default title page date field

\title{
    \thispagestyle{empty} % Suppress headers and footers
    \vspace{0.2\textheight} % Whitespace before the title
    \textbf{\assignmentClass}\\[5pt]
    \texttt{\assignmentTitle}\\[-4pt]
    % \ifdef{\assignmentSubTitle}{\texttt{\assignmentSubTitle}}{}
    \ifdef{\assignmentDate}{\assignmentDate}{} % If a due date is supplied, output it
    \ifdef{\assignmentClassInstructor}{{\large \textit{\assignmentClassInstructor}}}{} % If an instructor is supplied, output it
    \vspace{0.32\textheight} % Whitespace before the author name
}


\newcommand{\assignmentQuestionName}{Question}
\newcommand{\assignmentClass}{计算方法B}
\newcommand{\assignmentTitle}{Homework\ \#12}
\newcommand{\assignmentDate}{2020.6.15}
\newcommand{\assignmentNo}{PB17000297}
\newcommand{\assignmentAuthorName}{罗晏宸}
\usepackage{xspace}

\begin{document}

\maketitle

\thispagestyle{empty}

\newpage

\begin{question}

    \questiontext
    {
        用幂法估算下面矩阵的按模最大的特征值和相应的特征向量(取初始向量 $(1,\,1)^\mathsf{T}$ ,迭代 5 次即可)
        \begin{equation*}
            A = \begin{pmatrix}
                1 & 2 \\
                6 & 0
            \end{pmatrix}
        \end{equation*}
    }
    \answer
    {
    取 $X^{(0)} = \begin{pmatrix}
            1 \\
            1
        \end{pmatrix}$,则有
    \begin{align*}
        X^{(1)} = A \cdot X^{(0)} & =
        \begin{pmatrix}
            3 \\
            6
        \end{pmatrix}     \\
        X^{(2)} = A \cdot X^{(1)} & =
        \begin{pmatrix}
            15 \\
            18
        \end{pmatrix}     \\
        X^{(3)} = A \cdot X^{(2)} & =
        \begin{pmatrix}
            51 \\
            90
        \end{pmatrix}     \\
        X^{(4)} = A \cdot X^{(3)} & =
        \begin{pmatrix}
            231 \\
            306
        \end{pmatrix}     \\
        X^{(5)} = A \cdot X^{(4)} & =
        \begin{pmatrix}
            843 \\
            1386
        \end{pmatrix}
    \end{align*}
    在下表中列出迭代序列 $X^{(0)},\, X^{(1)},\, \cdots,\, X^{(5)}$ 以及 $\left.x_1^{(k)} \middle/ x_1^{(k - 1)}\right.$ 和 $\left.x_2^{(k)} \middle/ x_2^{(k - 1)}\right.$ 的值
    \begin{center}
        \begin{tabular}{|c|c|c|c|c|}
            \hline
            $k$ & \multicolumn{2}{|c|}{$X^{(k)}$} & $\left.x_1^{(k)} \middle/ x_1^{(k - 1)}\right.$ & $\left.x_2^{(k)} \middle/ x_2^{(k - 1)}\right.$          \\ \hline
            0   & 1                               & 1                                               &                                                 &        \\ \hline
            1   & 3                               & 6                                               & 3.0000                                          & 6.0000 \\ \hline
            2   & 15                              & 18                                              & 5.0000                                          & 3.0000 \\ \hline
            3   & 51                              & 90                                              & 3.4000                                          & 5.0000 \\ \hline
            4   & 231                             & 306                                             & 4.5294                                          & 3.4000 \\ \hline
            5   & 843                             & 1386                                            & 3.6494                                          & 4.5294 \\ \hline
        \end{tabular}
    \end{center}
    则得到矩阵 $A$ 粗略估计的按模最大特征值 $\lambda = 3.6494$ ,相应的特征向量近似地为 $X^{(5)} = (843,\, 1386)^\mathsf{T}$.
    }
\end{question}

\begin{question}
    \questiontext
    {
        设 $n$ 阶实方阵 $A$ 有相异的特征根 $|\lambda_1| > |\lambda_2| > \cdots > |\lambda_n| > 0$ . 对给定的实数 $\alpha \neq \lambda_{i} (i = 1,2, \cdots, n)$ ,利用幂法或反幂法,设计一个能计算离 $\alpha$ 距离最近的矩阵 $A$ 的特征根的迭代格式(注:不容许对矩阵求逆)
    }
    \answer
    {
        采取带原点位移的反幂法规范迭代计算公式
        \begin{equation*}
            \left\{
            \begin{aligned}
                Y^{(k)} = \left.X^{(k)} \middle/ \Arrowvert X^{(k)} \Arrowvert_{\infty} \right. \\
                (A - \alpha \mathbf{I})X^{(k + 1)} = Y^{(k)}
            \end{aligned}
            \right. \qquad k = 0,\, 1,\, \cdots
        \end{equation*}
        计算得到矩阵 $(A - \alpha \mathbf{I})$ 按模最小的特征值的倒数 $\mu$ ,则所求离 $\alpha$ 距离最近的矩阵 $A$ 的特征值为
        \begin{equation*}
            \lambda_i = \alpha + \frac{1}{\mu}
        \end{equation*}
    }
\end{question}

\begin{question}
    \questiontext
    {
        考虑用 Jacobi 方法计算矩阵 $A = \begin{pmatrix}
                7 & 1 & 2 \\
                1 & 4 & 0 \\
                2 & 0 & 3
            \end{pmatrix}$ 的特征值。求对 $A$ 作一次 Givens 相似变换时的 Givens (旋转)变换矩阵 $Q$ (要求相应的计算效率最高)
    }
    \answer
    {
        记 $A^{(0)} = A$ ,计算效率最高选取 $p = 1,\ q = 3$ ,有 $a^{(0)}_{pq} = a^{(0)}_{13} = 2$ 是模最大的非对角元素,于是有
        $$
            s = \frac{a^{(0)}_{33} - a^{(0)}_{11}}{2a^{(0)}_{13}} = \frac{3 - 7}{2 \times 2} = -1
        $$
        $t$ 取为 $t^2 + 2st - 1 = 0$ 的按模较小根,故 $t = 1 - \sqrt{2} \approx -0.414$. 进而得到:
        \begin{align*}
            \cos\theta & = \left(1 + t^2\right)^{-\frac{1}{2}} = 0.92388 \\
            \sin\theta & = t \cos\theta = -0.382683
        \end{align*}
        此即为 Givens 变换矩阵所需元素.
        \begin{equation*}
            Q_1 = Q(1, 3, \theta) = \begin{pmatrix}
                \phantom{-}\cos\theta & 0 & \phantom{-}\sin\theta \\
                \phantom{-}0          & 1 & \phantom{-}0          \\
                -\sin\theta           & 0 & \phantom{-}\cos\theta
            \end{pmatrix} = \begin{pmatrix}
                \phantom{-}0.92388  & 0 & -0.382683          \\
                \phantom{-}0        & 1 & \phantom{-}0       \\
                \phantom{-}0.382683 & 0 & \phantom{-}0.92388
            \end{pmatrix}
        \end{equation*}
        故有
        \begin{equation*}
            A^{(1)} = Q_1^{\mathsf{T}}A^{(0)}Q_1 =
            \begin{pmatrix}
                \phantom{-}7.82843\phantom{ \times 10 ^{-16}} & \phantom{-}0.92388\phantom{ \times 10 ^{-16}} & -4.44089 \times 10 ^{-16}                     \\
                \phantom{-}0.92388\phantom{ \times 10 ^{-16}} & \phantom{-}4.00000\phantom{ \times 10 ^{-16}} & -0.382683\phantom{ \times 10 ^{-16}}          \\
                -4.44089 \times 10 ^{-16}                     & -0.382683\phantom{ \times 10 ^{-16}}          & \phantom{-}2.17157\phantom{ \times 10 ^{-16}} \\
            \end{pmatrix}
        \end{equation*}
    }
\end{question}
\end{document}
