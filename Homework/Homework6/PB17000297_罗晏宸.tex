\documentclass[11pt]{article}
\usepackage[UTF8]{ctex}

%%%%%%%%%%%%%%%%%%%%%%%%%%%%%%%%%%%%%%%%%
% Cleese Assignment
% Structure Specification File
% Version 1.0 (27/5/2018)
%
% This template originates from:
% http://www.LaTeXTemplates.com
%
% Author:
% Vel (vel@LaTeXTemplates.com)
%
% License:
% CC BY-NC-SA 3.0 (http://creativecommons.org/licenses/by-nc-sa/3.0/)
%
%%%%%%%%%%%%%%%%%%%%%%%%%%%%%%%%%%%%%%%%%

%----------------------------------------------------------------------------------------
%	PACKAGES AND OTHER DOCUMENT CONFIGURATIONS
%----------------------------------------------------------------------------------------

\usepackage{cite}

\usepackage{lastpage} % Required to determine the last page number for the footer

\usepackage{graphicx} % Required to insert images

\setlength\parindent{0pt} % Removes all indentation from paragraphs

\usepackage[most]{tcolorbox} % Required for boxes that split across pages

\usepackage{booktabs} % Required for better horizontal rules in tables

\usepackage{listings} % Required for insertion of code

\usepackage{etoolbox} % Required for if statements

\usepackage{amsmath}
% 增加矩阵两侧括号内侧的空白,新建附参数arraystretch的matrix命令族
\makeatletter
\renewenvironment{pmatrix}
{\left(\mkern10.0mu\env@matrix}
{\endmatrix\mkern10.0mu\right)}
\renewenvironment{vmatrix}
{\left|\mkern10.0mu\env@matrix}
{\endmatrix\mkern10.0mu\right|}
\renewcommand*\env@matrix[1][\arraystretch]{%
    \edef\arraystretch{#1}%
    \hskip -\arraycolsep
    \let\@ifnextchar\new@ifnextchar
    \array{*\c@MaxMatrixCols c}}
\makeatother

\usepackage{amsthm}
\usepackage{amssymb}
\usepackage{indentfirst}
\usepackage{diagbox}
\usepackage{float}
\usepackage{xcolor}
\usepackage[colorlinks, linkcolor = black]{hyperref}

\usepackage{enumerate}
\usepackage{enumitem}
\setlist{
    leftmargin = .1\linewidth,
    % rightmargin = .1\linewidth,
    % label=\emph{\alph*}.
}

\setlength{\parindent}{2em}

\usepackage{siunitx}
\sisetup
{
    output-exponent-marker = \ensuremath{\mathrm{E}},
    exponent-product = {},
    retain-explicit-plus,
    retain-zero-exponent,
}
%----------------------------------------------------------------------------------------
%	MARGINS
%----------------------------------------------------------------------------------------

\usepackage{geometry} % Required for adjusting page dimensions and margins

\geometry{
    paper=a4paper, % Change to letterpaper for US letter
    top=3cm, % Top margin
    bottom=3cm, % Bottom margin
    left=2.5cm, % Left margin
    right=2.5cm, % Right margin
    headheight=14pt, % Header height
    footskip=1.4cm, % Space from the bottom margin to the baseline of the footer
    headsep=1.2cm, % Space from the top margin to the baseline of the header
    %showframe, % Uncomment to show how the type block is set on the page
}

%----------------------------------------------------------------------------------------
%	FONT
%----------------------------------------------------------------------------------------

\usepackage[utf8]{inputenc} % Required for inputting international characters
\usepackage[T1]{fontenc} % Output font encoding for international characters

% \usepackage[sfdefault,light]{roboto} % Use the Roboto font

%----------------------------------------------------------------------------------------
%	HEADERS AND FOOTERS
%----------------------------------------------------------------------------------------

\usepackage{fancyhdr} % Required for customising headers and footers

\pagestyle{fancy} % Enable custom headers and footers

\lhead{\small\assignmentClass} % Left header; output the instructor in brackets if one was set
\chead{\small\assignmentTitle} % Centre header
\rhead{\small\ifdef{\assignmentAuthorName}{\assignmentAuthorName}{\ifdef{\assignmentDate}{Due\ \assignmentDate}{}}} % Right header; output the author name if one was set, otherwise the due date if that was set

\lfoot{} % Left footer
\cfoot{\small Page\ \thepage\ of\ \pageref{LastPage}} % Centre footer
\rfoot{} % Right footer

\renewcommand\headrulewidth{0.5pt} % Thickness of the header rule

%----------------------------------------------------------------------------------------
%	MODIFY SECTION STYLES
%----------------------------------------------------------------------------------------

\usepackage{titlesec} % Required for modifying sections

%------------------------------------------------
% Section

\titleformat
{\section} % Section type being modified
[block] % Shape type, can be: hang, block, display, runin, leftmargin, rightmargin, drop, wrap, frame
{\Large\bfseries} % Format of the whole section
{\arabic{section}} % Format of the section label
{6pt} % Space between the title and label
{} % Code before the label

\titlespacing{\section}{0pt}{0.5\baselineskip}{0.5\baselineskip} % Spacing around section titles, the order is: left, before and after

%------------------------------------------------
% Subsection

\titleformat
{\subsection} % Section type being modified
[hang] % Shape type, can be: hang, block, display, runin, leftmargin, rightmargin, drop, wrap, frame
{\large\bfseries} % Format of the whole section
{\arabic{section}.\arabic{subsection}} % Format of the section label
{4pt} % Space between the title and label
{} % Code before the label

\titlespacing{\subsection}{0pt}{0.5\baselineskip}{0.5\baselineskip} % Spacing around section titles, the order is: left, before and after

\renewcommand\thesubsection{\arabic{section}.\arabic{subsection}}

%----------------------------------------------------------------------------------------
%	CUSTOM QUESTION COMMANDS/ENVIRONMENTS
%----------------------------------------------------------------------------------------



% Command to print an assignment section title to split an assignment into major parts
\newcommand{\assignmentSection}[1]{
    \newpage
    {
        \centering % Centre the section title
        \vspace{2\baselineskip} % Whitespace before the entire section title

        \rule{0.8\textwidth}{0.5pt} % Horizontal rule

        \vspace{0.75\baselineskip} % Whitespace before the section title
        {\LARGE \textsc{#1}} % Section title, forced to be uppercase

        \rule{0.8\textwidth}{0.5pt} % Horizontal rule

        \vspace{\baselineskip} % Whitespace after the entire section title
    }
    \setcounter{section}{0}

}

%----------------------------------------------------------------------------------------
%	TITLE PAGE
%----------------------------------------------------------------------------------------

\author{\textbf{\assignmentNo\ \assignmentAuthorName}} % Set the default title page author field
\date{} % Don't use the default title page date field

\title{
    \thispagestyle{empty} % Suppress headers and footers
    \vspace{0.2\textheight} % Whitespace before the title
    \textbf{\assignmentClass}\\[5pt]
    \texttt{\assignmentTitle}\\[-4pt]
    % \ifdef{\assignmentSubTitle}{\texttt{\assignmentSubTitle}}{}
    \ifdef{\assignmentDate}{\assignmentDate}{} % If a due date is supplied, output it
    \ifdef{\assignmentClassInstructor}{{\large \textit{\assignmentClassInstructor}}}{} % If an instructor is supplied, output it
    \vspace{0.32\textheight} % Whitespace before the author name
}


\newcommand{\assignmentQuestionName}{Question}
\newcommand{\assignmentClass}{计算方法B}
\newcommand{\assignmentTitle}{Homework\ \#6}
\newcommand{\assignmentDate}{2020.5.1}
\newcommand{\assignmentNo}{PB17000297}
\newcommand{\assignmentAuthorName}{罗晏宸}

\begin{document}

\maketitle

\thispagestyle{empty}

\newpage

\begin{question}

    \questiontext
    {
        分别计算下列矩阵的$\Arrowvert \bullet \Arrowvert_1$,$\Arrowvert \bullet \Arrowvert_\infty$范数:
    }
    \begin{subquestion}{}
        \questiontext
        {
            \begin{equation*}
                A = \begin{pmatrix}
                    \phantom{-}5 & -2           & \phantom{-}1 \\
                    -1           & \phantom{-}3 & -1           \\
                    -1           & \phantom{-}2 & -6
                \end{pmatrix}
            \end{equation*}
        }
        \answer
        {
            \begin{align*}
                \Arrowvert A \Arrowvert_1      & = \max_{1 \leqslant j \leqslant 3}{\left\{\sum_{i = 1}^{3}{\arrowvert a_{ij} \arrowvert}\right\}} \\
                                               & = \max{\{7,\, 7,\, 8\}}                                                                           \\
                                               & = 8                                                                                               \\
                \Arrowvert A \Arrowvert_\infty & = \max_{1 \leqslant i \leqslant 3}{\left\{\sum_{j = 1}^{3}{\arrowvert a_{ij} \arrowvert}\right\}} \\
                                               & = \max{\{8,\, 5,\, 9\}}                                                                           \\
                                               & = 9
            \end{align*}
        }
    \end{subquestion}
    \begin{subquestion}{}
        \questiontext
        {
            \begin{equation*}
                A = \begin{pmatrix}
                    \phantom{-}4 & -1           & \phantom{-}1 \\
                    -2           & \phantom{-}3 & \phantom{-}1 \\
                    -1           & \phantom{-}1 & \phantom{-}6
                \end{pmatrix}
            \end{equation*}
        }
        \answer
        {
            \begin{align*}
                \Arrowvert A \Arrowvert_1      & = \max_{1 \leqslant j \leqslant 3}{\left\{\sum_{i = 1}^{3}{\arrowvert a_{ij} \arrowvert}\right\}} \\
                                               & = \max{\{7,\, 5,\, 8\}}                                                                           \\
                                               & = 8                                                                                               \\
                \Arrowvert A \Arrowvert_\infty & = \max_{1 \leqslant i \leqslant 3}{\left\{\sum_{j = 1}^{3}{\arrowvert a_{ij} \arrowvert}\right\}} \\
                                               & = \max{\{6,\, 6,\, 8\}}                                                                           \\
                                               & = 8
            \end{align*}
        }
    \end{subquestion}

\end{question}

\begin{question}

    \questiontext
    {
        分别计算矩阵$A =
            \begin{pmatrix}
                \phantom{-}4 & -2           \\
                -1           & \phantom{-}1 \\
            \end{pmatrix}$
        的谱半径及其$\Arrowvert \bullet \Arrowvert_2$范数。
    }

    \answer
    {
        设$\lambda$为 $A$ 的特征值,则有
        \begin{align*}
                        &  & \det{(A - \lambda \mathbf{I})}                & = 0 &  &  & \\
            \Rightarrow &  & \begin{vmatrix}
                4 - \lambda & -2          \\
                -1          & 1 - \lambda \\
            \end{vmatrix}                     & = 0 &  &  & \\
            \Rightarrow &  & (4 - \lambda)(1 - \lambda) - (-1) \times (-2) & = 0 &  &  & \\
            \Rightarrow &  & \lambda^2 - 5 \lambda + 2                     & = 0 &  &  &
        \end{align*}
        解得
        \begin{equation*}
            \lambda_1 = \frac{5-\sqrt{17}}{2},\ \lambda_2 = \frac{5+\sqrt{17}}{2}
        \end{equation*}
        进而有$A$的谱半径
        \begin{align*}
            \rho(A) & = \max_i{|\lambda_i|}                                  \\
                    & = \max{\frac{5-\sqrt{17}}{2},\, \frac{5+\sqrt{17}}{2}} \\
                    & = \frac{5+\sqrt{17}}{2}
        \end{align*}
        设$\mu$是矩阵$A^{\mathbf{T}}\!A$的特征值,则有
        \begin{align*}
                        &  & \det(A^{\mathbf{T}}\!A - \mu\mathbf{I})                                                             & = 0 &  &  & \\
            \Rightarrow &  & \det{\left(\begin{pmatrix}
                    \phantom{-}4 & -1           \\
                    -2           & \phantom{-}1 \\
                \end{pmatrix}\begin{pmatrix}
                    \phantom{-}4 & -2           \\
                    -1           & \phantom{-}1 \\
                \end{pmatrix} - \begin{pmatrix}
                    \mu & 0   \\
                    0   & \mu \\
                \end{pmatrix}\right)} & = 0 &  &  & \\
            \Rightarrow &  & \begin{vmatrix}
                17 - \mu & -9     \\
                -9       & 5 -\mu \\
            \end{vmatrix}                                                                          & = 0 &  &  & \\
            \Rightarrow &  & (5 -\mu)(17 -\mu) - (-9) \times (-9)                                                                & = 0 &  &  & \\
            \Rightarrow &  & \mu^2 - 22\mu + 24                                                                                  & = 0 &  &  & \\
        \end{align*}
        解得
        \begin{equation*}
            \mu_1 = 11-3 \sqrt{13},\ \mu_2 = 11+3 \sqrt{13}
        \end{equation*}
        由此矩阵$A$的$\Arrowvert \bullet \Arrowvert_2$范数计算如下
        \begin{align*}
            \Arrowvert A \Arrowvert_2 & = \sqrt{\max_{1 \leqslant i \leqslant n}{\{\left|\mu_i\right|}\}} \\
                                      & = \sqrt{\max{\{11-3 \sqrt{13},\, 11+3 \sqrt{13}\}}}               \\
                                      & = \sqrt{11+3 \sqrt{13}}
        \end{align*}
    }

\end{question}

\begin{question}

    \questiontext
    {
        用 Doolittle 分解法解下列线性方程组(请给出详细的解题过程,包括矩阵分解)
        \begin{equation*}
            \left\{
            \begin{aligned}
                 & 5x_1 + \phantom{1}x_2 + 2x_3           &  & = 10 & \\
                 & \phantom{1}x_1 + 3x_2 - \phantom{1}x_3 &  & = 5  & \\
                 & 2x_1 + \phantom{1}x_2 + 5x_3           &  & = 20 & \\
            \end{aligned}
            \right.
        \end{equation*}
    }

    \answer
    {
        以矩阵形式表示题设线性方程组为$Ax = b$,其中
        \begin{equation*}
            A = \begin{pmatrix}
                5 & 1 & \phantom{-}2 \\
                1 & 3 & -1           \\
                2 & 1 & \phantom{-}5
            \end{pmatrix}
            , b = \begin{pmatrix}
                10 \\
                5  \\
                20
            \end{pmatrix}
        \end{equation*}
        设
        \begin{equation*}
            A = LU = \begin{pmatrix}
                1      & 0      & 0 \\
                l_{21} & 1      & 0 \\
                l_{31} & l_{32} & 1
            \end{pmatrix} \begin{pmatrix}
                u_{11} & u_{12} & u_{13} \\
                0      & u_{22} & u_{23} \\
                0      & 0      & u_{33}
            \end{pmatrix}
        \end{equation*}

        首先计算$U$的第一行元素有
        \begin{equation*}
            a_{1j} = \sum_{r = 1}^3{l_{1r}u_{rj}} =
            \begin{pmatrix}
                1 & 0 & 0
            \end{pmatrix}
            \begin{pmatrix}
                u_{13} \\
                u_{23} \\
                u_{33}
            \end{pmatrix}
            = u_{1j},\quad j = 1, 2, 3
        \end{equation*}
        得到
        \begin{equation*}
            u_{11} = a_{11} = 5,\
            u_{12} = a_{12} = 1,\
            u_{13} = a_{13} = 2,\
        \end{equation*}

        其次计算$L$的第一列元素有
        \begin{equation*}
            a_{i1} = \sum_{r = 1}^3{l_{ir}u_{r1}} =
            \begin{pmatrix}
                l_{i1} & \cdots & l_{i,i-1} & 1 & 0 & \cdots & 0
            \end{pmatrix}
            \begin{pmatrix}
                u_{11} \\
                0      \\
                \vdots \\
                0
            \end{pmatrix}
            = l_{i1}u_{11},\quad i = 2, 3
        \end{equation*}
        得到
        \begin{equation*}
            l_{21} = \frac{a_{21}}{u_{11}} = \frac{1}{5},\
            l_{31} = \frac{a_{31}}{u_{11}} = \frac{2}{5}
        \end{equation*}

        然后由
        \begin{equation*}
            a_{2i} =
            \begin{pmatrix}
                l_{21} & 1 & 0
            \end{pmatrix}
            \begin{pmatrix}
                u_{1i} \\
                u_{2i} \\
                u_{3i}
            \end{pmatrix}
            = l_{21}u_{1i} + u_{2i},\quad i = 2,3
        \end{equation*}
        得到
        \begin{equation*}
            u_{22} = a_{22} - l_{21}u_{12} = 3 - \frac{1}{5} \times 1 = \frac{14}{5},\ u_{23} = a_{23} - l_{21}u_{13} = -1 - \frac{1}{5} \times 2 = -\frac{7}{5}
        \end{equation*}

        再由
        \begin{equation*}
            a_{32} =
            \begin{pmatrix}
                l_{31} & l_{32} & 1
            \end{pmatrix}
            \begin{pmatrix}
                u_{12} \\
                u_{22} \\
                0
            \end{pmatrix}
            = l_{31}u_{12} + l_{32}u_{22}
        \end{equation*}
        得到
        \begin{equation*}
            l_{32} = \frac{a_{32} - l_{31}u_{12}}{u_{22}} = \frac{1 - \dfrac{2}{5} \times 1}{\dfrac{14}{5}} = \frac{3}{14}
        \end{equation*}

        最后由
        \begin{equation*}
            a_{33} =
            \begin{pmatrix}
                l_{31} & l_{32} & 1
            \end{pmatrix}
            \begin{pmatrix}
                u_{13} \\
                u_{23} \\
                u_{33}
            \end{pmatrix}
            = l_{31}u_{13} + l_{32}u_{23} + u_{33}
        \end{equation*}
        得到
        \begin{equation*}
            u_{33} = a_{33} - l_{31}u_{13} - l_{32}u_{23} = 5 - \frac{2}{5} \times 2 - \frac{3}{14} \times (-\frac{7}{5}) = \frac{9}{2}
        \end{equation*}

        综上得到
        \begin{equation*}
            A = LU = \begin{pmatrix}[1.7]
                1            & 0             & 0 \\
                \dfrac{1}{5} & 1             & 0 \\
                \dfrac{2}{5} & \dfrac{3}{14} & 1
            \end{pmatrix} \begin{pmatrix}[1.7]
                5 & 1             & \phantom{-}2            \\
                0 & \dfrac{14}{5} & -\dfrac{7}{5}           \\
                0 & 0             & \phantom{-}\dfrac{9}{2}
            \end{pmatrix}
        \end{equation*}

        解方程$Ly = b$
        \begin{align*}
            \begin{pmatrix}[1.7]
                1            & 0             & 0 \\
                \dfrac{1}{5} & 1             & 0 \\
                \dfrac{2}{5} & \dfrac{3}{14} & 1
            \end{pmatrix}
            \begin{pmatrix}[1.7]
                y_1 \\
                y_2 \\
                y_3
            \end{pmatrix}
             & =
            \begin{pmatrix}[1.7]
                10 \\
                5  \\
                20
            \end{pmatrix} \\
            \begin{pmatrix}[1.7]
                y_1 \\
                y_2 \\
                y_3
            \end{pmatrix}
             & =
            \begin{pmatrix}[1.7]
                10 \\
                3  \\
                \dfrac{215}{14}
            \end{pmatrix}
        \end{align*}

        解方程$Ux = y$
        \begin{align*}
            \begin{pmatrix}[1.7]
                5 & 1             & \phantom{-}2            \\
                0 & \dfrac{14}{5} & -\dfrac{7}{5}           \\
                0 & 0             & \phantom{-}\dfrac{9}{2}
            \end{pmatrix}
            \begin{pmatrix}[1.7]
                x_1 \\
                x_2 \\
                x_3
            \end{pmatrix}
             & =
            \begin{pmatrix}[1.7]
                10 \\
                3  \\
                \dfrac{215}{14}
            \end{pmatrix} \\
            \begin{pmatrix}[1.7]
                x_1 \\
                x_2 \\
                x_3
            \end{pmatrix}
             & =
            \begin{pmatrix}[1.7]
                \dfrac{5}{63} \\
                \dfrac{25}{9} \\
                \dfrac{215}{63}
            \end{pmatrix}
            \approx
            \begin{pmatrix}[1.7]
                0.0794 \\
                2.7778 \\
                3.4127
            \end{pmatrix}
        \end{align*}
    }

\end{question}
\end{document}
