\documentclass[11pt]{article}
\usepackage[UTF8]{ctex}

%%%%%%%%%%%%%%%%%%%%%%%%%%%%%%%%%%%%%%%%%
% Cleese Assignment
% Structure Specification File
% Version 1.0 (27/5/2018)
%
% This template originates from:
% http://www.LaTeXTemplates.com
%
% Author:
% Vel (vel@LaTeXTemplates.com)
%
% License:
% CC BY-NC-SA 3.0 (http://creativecommons.org/licenses/by-nc-sa/3.0/)
%
%%%%%%%%%%%%%%%%%%%%%%%%%%%%%%%%%%%%%%%%%

%----------------------------------------------------------------------------------------
%	PACKAGES AND OTHER DOCUMENT CONFIGURATIONS
%----------------------------------------------------------------------------------------

\usepackage{cite}

\usepackage{lastpage} % Required to determine the last page number for the footer

\usepackage{graphicx} % Required to insert images

\setlength\parindent{0pt} % Removes all indentation from paragraphs

\usepackage[most]{tcolorbox} % Required for boxes that split across pages

\usepackage{booktabs} % Required for better horizontal rules in tables

\usepackage{listings} % Required for insertion of code

\usepackage{etoolbox} % Required for if statements

\usepackage{amsmath}
% 增加矩阵两侧括号内侧的空白,新建附参数arraystretch的matrix命令族
\makeatletter
\renewenvironment{pmatrix}
{\left(\mkern10.0mu\env@matrix}
{\endmatrix\mkern10.0mu\right)}
\renewenvironment{vmatrix}
{\left|\mkern10.0mu\env@matrix}
{\endmatrix\mkern10.0mu\right|}
\renewcommand*\env@matrix[1][\arraystretch]{%
    \edef\arraystretch{#1}%
    \hskip -\arraycolsep
    \let\@ifnextchar\new@ifnextchar
    \array{*\c@MaxMatrixCols c}}
\makeatother

\usepackage{amsthm}
\usepackage{amssymb}
\usepackage{indentfirst}
\usepackage{diagbox}
\usepackage{float}
\usepackage{xcolor}
\usepackage[colorlinks, linkcolor = black]{hyperref}

\usepackage{enumerate}
\usepackage{enumitem}
\setlist{
    leftmargin = .1\linewidth,
    % rightmargin = .1\linewidth,
    % label=\emph{\alph*}.
}

\setlength{\parindent}{2em}

\usepackage{siunitx}
\sisetup
{
    output-exponent-marker = \ensuremath{\mathrm{E}},
    exponent-product = {},
    retain-explicit-plus,
    retain-zero-exponent,
}
%----------------------------------------------------------------------------------------
%	MARGINS
%----------------------------------------------------------------------------------------

\usepackage{geometry} % Required for adjusting page dimensions and margins

\geometry{
    paper=a4paper, % Change to letterpaper for US letter
    top=3cm, % Top margin
    bottom=3cm, % Bottom margin
    left=2.5cm, % Left margin
    right=2.5cm, % Right margin
    headheight=14pt, % Header height
    footskip=1.4cm, % Space from the bottom margin to the baseline of the footer
    headsep=1.2cm, % Space from the top margin to the baseline of the header
    %showframe, % Uncomment to show how the type block is set on the page
}

%----------------------------------------------------------------------------------------
%	FONT
%----------------------------------------------------------------------------------------

\usepackage[utf8]{inputenc} % Required for inputting international characters
\usepackage[T1]{fontenc} % Output font encoding for international characters

% \usepackage[sfdefault,light]{roboto} % Use the Roboto font

%----------------------------------------------------------------------------------------
%	HEADERS AND FOOTERS
%----------------------------------------------------------------------------------------

\usepackage{fancyhdr} % Required for customising headers and footers

\pagestyle{fancy} % Enable custom headers and footers

\lhead{\small\assignmentClass} % Left header; output the instructor in brackets if one was set
\chead{\small\assignmentTitle} % Centre header
\rhead{\small\ifdef{\assignmentAuthorName}{\assignmentAuthorName}{\ifdef{\assignmentDate}{Due\ \assignmentDate}{}}} % Right header; output the author name if one was set, otherwise the due date if that was set

\lfoot{} % Left footer
\cfoot{\small Page\ \thepage\ of\ \pageref{LastPage}} % Centre footer
\rfoot{} % Right footer

\renewcommand\headrulewidth{0.5pt} % Thickness of the header rule

%----------------------------------------------------------------------------------------
%	MODIFY SECTION STYLES
%----------------------------------------------------------------------------------------

\usepackage{titlesec} % Required for modifying sections

%------------------------------------------------
% Section

\titleformat
{\section} % Section type being modified
[block] % Shape type, can be: hang, block, display, runin, leftmargin, rightmargin, drop, wrap, frame
{\Large\bfseries} % Format of the whole section
{\arabic{section}} % Format of the section label
{6pt} % Space between the title and label
{} % Code before the label

\titlespacing{\section}{0pt}{0.5\baselineskip}{0.5\baselineskip} % Spacing around section titles, the order is: left, before and after

%------------------------------------------------
% Subsection

\titleformat
{\subsection} % Section type being modified
[hang] % Shape type, can be: hang, block, display, runin, leftmargin, rightmargin, drop, wrap, frame
{\large\bfseries} % Format of the whole section
{\arabic{section}.\arabic{subsection}} % Format of the section label
{4pt} % Space between the title and label
{} % Code before the label

\titlespacing{\subsection}{0pt}{0.5\baselineskip}{0.5\baselineskip} % Spacing around section titles, the order is: left, before and after

\renewcommand\thesubsection{\arabic{section}.\arabic{subsection}}

%----------------------------------------------------------------------------------------
%	CUSTOM QUESTION COMMANDS/ENVIRONMENTS
%----------------------------------------------------------------------------------------



% Command to print an assignment section title to split an assignment into major parts
\newcommand{\assignmentSection}[1]{
    \newpage
    {
        \centering % Centre the section title
        \vspace{2\baselineskip} % Whitespace before the entire section title

        \rule{0.8\textwidth}{0.5pt} % Horizontal rule

        \vspace{0.75\baselineskip} % Whitespace before the section title
        {\LARGE \textsc{#1}} % Section title, forced to be uppercase

        \rule{0.8\textwidth}{0.5pt} % Horizontal rule

        \vspace{\baselineskip} % Whitespace after the entire section title
    }
    \setcounter{section}{0}

}

%----------------------------------------------------------------------------------------
%	TITLE PAGE
%----------------------------------------------------------------------------------------

\author{\textbf{\assignmentNo\ \assignmentAuthorName}} % Set the default title page author field
\date{} % Don't use the default title page date field

\title{
    \thispagestyle{empty} % Suppress headers and footers
    \vspace{0.2\textheight} % Whitespace before the title
    \textbf{\assignmentClass}\\[5pt]
    \texttt{\assignmentTitle}\\[-4pt]
    % \ifdef{\assignmentSubTitle}{\texttt{\assignmentSubTitle}}{}
    \ifdef{\assignmentDate}{\assignmentDate}{} % If a due date is supplied, output it
    \ifdef{\assignmentClassInstructor}{{\large \textit{\assignmentClassInstructor}}}{} % If an instructor is supplied, output it
    \vspace{0.32\textheight} % Whitespace before the author name
}


\newcommand{\assignmentQuestionName}{Question}
\newcommand{\assignmentClass}{计算方法B}
\newcommand{\assignmentTitle}{Homework\ \#3}
\newcommand{\assignmentDate}{2020.4.13}
\newcommand{\assignmentNo}{PB17000297}
\newcommand{\assignmentAuthorName}{罗晏宸}

\begin{document}

\maketitle

\thispagestyle{empty}

\newpage

\begin{question}

	\questiontext
	{
		求满足下表数据以及边界条件$S''(-2) = S''(2) = 0\ (n = 3)$的三次样条插值函数$S(x)$,并计算$S(0)$的值。注意:这里的$n$为小区间个数。
		\begin{table}[h]
			\centering
			\begin{tabular}{|c|c|c|c|c|}
				\hline
				$x$    & $-2.00$ & $-1.00$ & $1.00$ & $2.00$ \\ \hline
				$f(x)$ & $-4.00$ & $3.00$  & $5.00$ & $10.0$ \\ \hline
			\end{tabular}
		\end{table}
	}

	\answer{
		在三个子区间上分别构造三次多项式
		\begin{equation*}
			S(x) = \left\{
			\begin{aligned}
				S_0(x) & = a_0x^3 + b_0x^2 + c_0x + d_0, &  & x \in [-2.00, -1.00] \\
				S_1(x) & = a_1x^3 + b_1x^2 + c_1x + d_1, &  & x \in [-1.00,  1.00] \\
				S_2(x) & = a_2x^3 + b_2x^2 + c_2x + d_2, &  & x \in [ 1.00,  2.00]
			\end{aligned}
			\right.
		\end{equation*}
		共有12个未知数,需要等量的条件。由表中数据、插值函数在每个内点的关系以及边界条件有
		\begin{align*}
			\left\{
			\begin{aligned}
				S_0(-2.00)   & = -4.00        \\
				S_0(-1.00)   & = 3.00         \\
				S_1(-1.00)   & = 3.00         \\
				S_1(1.00)    & = 5.00         \\
				S_2(1.00)    & = 5.00         \\
				S_2(2.00)    & = 10.00        \\
				S'_0(-1.00)  & = S'_1(-1.00)  \\
				S'_1(1.00)   & = S'_2(1.00)   \\
				S''_0(-1.00) & = S''_1(-1.00) \\
				S''_1(1.00)  & = S''_2(1.00)  \\
				S''_0(-2.00) & = 0            \\
				S''_2(2.00)  & = 0
			\end{aligned}
			\right.
		\end{align*}
		\begin{align*}
			\Rightarrow\left\{
			\begin{aligned}
				(-2.00)^3a_0 + (-2.00)^2b_0 -2.00c_0 + d_0        & = -4.00                                              \\
				(-1.00)^3a_0 + (-1.00)^2b_0 -1.00c_0 + d_0        & = 3.00                                               \\
				(-1.00)^3a_1 + (-1.00)^2b_1 -1.00c_1 + d_1        & = 3.00                                               \\
				1.00^3a_1 + 1.00^2b_1 + 1.00c_1 + d_1             & = 5.00                                               \\
				1.00^3a_2 + 1.00^2b_2 + 1.00c_2 + d_2             & = 5.00                                               \\
				2.00^3a_2 + 2.00^2b_2 + 2.00c_2 + d_2             & = 10.00                                              \\
				3 \times (-1.00)^2a_0 + 2 \times (-1.00)b_0 + c_0 & =  3 \times (-1.00)^2a_1 + 2 \times (-1.00)b_1 + c_1 \\
				3 \times 1.00^2a_1 + 2 \times 1.00b_1 + c_1       & =  3 \times 1.00^2a_2 + 2 \times 1.00b_2 + c_2       \\
				6 \times(-1.00)a_0 + 2b_0                         & = 6 \times(-1.00)a_1 + 2b_1                          \\
				6 \times 1.00a_1 + 2b_1                           & = 6 \times 1.00a_2 + 2b_2                            \\
				6 \times (-2.00)a_0 + 2b_0                        & = 0                                                  \\
				6 \times 2.00a_2 + 2b_2                           & = 0
			\end{aligned}
			\right.
		\end{align*}
		\begin{align*}
			\Rightarrow\left\{
			\begin{aligned}
				-8a_0 + 4b_0 -2c_0 + d_0 & = -4                 \\
				-a_0 + b_0 - c_0 + d_0   & = 3                  \\
				-a_1 + b_1 - c_1 + d_1   & = 3                  \\
				a_1 + b_1 + c_1 + d_1    & = 5                  \\
				a_2 + b_2 + c_2 + d_2    & = 5                  \\
				8a_2 + 4b_2 + 2c_2 + d_2 & = 10                 \\
				3a_0 - 2b_0 + c_0        & =  3a_1 - 2b_1 + c_1 \\
				3a_1 + 2b_1 + c_1        & =  3a_2 + 2b_2 + c_2 \\
				-6a_0 + 2b_0             & = -6a_1 + 2b_1       \\
				6a_1 + 2b_1              & = 6a_2 + 2b_2        \\
				-12a_0 + 2b_0            & = 0                  \\
				12a_2 + 2b_2             & = 0
			\end{aligned}
			\right.
		\end{align*}
		\begin{align*}
			\Rightarrow\left\{
			\begin{aligned}
				a_0 & = -\frac{11}{8} \\
				b_0 & = -\frac{33}{4} \\
				c_0 & = -\frac{65}{8} \\
				d_0 & =  \frac{7}{4}  \\
				a_1 & =  \frac{5}{4}  \\
				b_1 & = -\frac{3}{8}  \\
				c_1 & = -\frac{1}{4}  \\
				d_1 & =  \frac{35}{8} \\
				a_2 & = -\frac{9}{8}  \\
				b_2 & =  \frac{27}{4} \\
				c_2 & = -\frac{59}{8} \\
				d_2 & =  \frac{27}{4}
			\end{aligned}
			\right.
		\end{align*}
		得到
		\begin{equation*}
			S(x) = \left\{
			\begin{aligned}
				S_0(x) & = & - \frac{11}{8} & x^3 \ - & \frac{33}{4} & x^2 \ - & \frac{65}{8} & x \ + & \frac{7}{4},   &  & x \in [-2.00, -1.00] \\
				S_1(x) & = & \frac{5}{4}    & x^3 \ - & \frac{3}{8}  & x^2 \ - & \frac{1}{4}  & x \ + & \frac{35}{8} , &  & x \in [-1.00,  1.00] \\
				S_2(x) & = & - \frac{9}{8}  & x^3 \ + & \frac{27}{4} & x^2 \ - & \frac{59}{8} & x \ + & \frac{27}{4} , &  & x \in [ 1.00,  2.00]
			\end{aligned}
			\right.
		\end{equation*}
		同时
		\begin{equation*}
			S(0) = S_1(0) = \frac{35}{8} = 4.375
		\end{equation*}
	}
\end{question}

\begin{question}

	\questiontext{利用下面的函数值表,构造分段线性插值函数,并计算$f(1.075)$和$f(1.175)$的近似值(保留4位小数)。
		\begin{table}[h]
			\centering
			\begin{tabular}{|c|c|c|c|c|}
				\hline
				$x$    & $1.05$ & $1.10$ & $1.15$ & $1.20$ \\ \hline
				$f(x)$ & $2.00$ & $2.20$ & $2.17$ & $2.35$ \\ \hline
			\end{tabular}
		\end{table}
	}

	\answer{
		在三个子区间上作$f(x)$以区间端点为节点的线性插值,有
		\begin{align*}
			p(x) & = \left\{
			\begin{aligned}
				p_0(x) & = \frac{x - 1.10}{1.05 - 1.10} \times 2.00 + \frac{x - 1.05}{1.10 - 1.05} \times 2.20, &  & x \in [1.05, 1.10] \\
				p_1(x) & = \frac{x - 1.15}{1.10 - 1.15} \times 2.20 + \frac{x - 1.10}{1.15 - 1.10} \times 2.17, &  & x \in [1.05, 1.10] \\
				p_2(x) & = \frac{x - 1.20}{1.15 - 1.20} \times 2.17 + \frac{x - 1.15}{1.20 - 1.15} \times 2.35, &  & x \in [1.05, 1.10]
			\end{aligned}
			\right.          \\
			     & = \left\{
			\begin{aligned}
				p_0(x) & = & 4.00  & x - 2.20, &  & x \in [1.05, 1.10] \\
				p_1(x) & = & -0.60 & x + 2.86, &  & x \in [1.10, 1.15] \\
				p_2(x) & = & 3.60  & x - 1.97, &  & x \in [1.15, 1.20]
			\end{aligned}
			\right.
		\end{align*}
		计算近似值有
		\begin{align*}
			f(1.075) & \approx p_0(1.075)  = 4.00 \times 1.075 - 2.20  = 2.1000 \\
			f(1.175) & \approx p_2(1.175)  = 3.60 \times 1.175 - 1.97  = 2.2600
		\end{align*}
	}
\end{question}

\begin{question}

	\questiontext{设$f(x) = 10x^3 + 3x + 2020$,求$f[1, 2]$和$f[1,2,3,4]$。}
	\answer{
		\begin{align*}
			f[1,2]        & = \frac{f(2) - f(1)}{2 - 1}                                                 \\
			              & = (10 \times 2^3 + 3 \times 2 + 2020) - (10 \times 1^3 + 3 \times 1 + 2020) \\
			              & = 73                                                                        \\
			              &                                                                             \\
			f[1, 2, 3, 4] & = \frac{f^{(3)}(\xi)}{3!},\quad \xi \in [1, 4]                              \\
			              & = \frac{60}{6}                                                              \\
			              & = 10
		\end{align*}
	}

\end{question}

\begin{question}

	\questiontext{设$\{l_i(x)\}_{i = 0}^6$是以$\{x_i = 2i\}_{i = 0}^6$为结点的6次 Lagrange 插值基函数,试求$\displaystyle \sum_{i = 0}^6{(x_i^3 + x_i^2 + 1)l_i(x)}$和$\displaystyle \sum_{i = 0}^6{(x_i^3 + x_i^2 + 1)l'_i(x)}$(结果需化简)。}

	\answer{
	形式上$$\sum_{i = 0}^6{(x_i^3 + x_i^2 + 1)l_i(x)}$$给出了函数$x^3 + x^2 + 1$以$\{x_i = 2i\}_{i = 0}^6$为插值结点横坐标的6次插值多项式,由插值多项式的存在唯一性有
	\begin{equation*}
		\sum_{i = 0}^6{(x_i^3 + x_i^2 + 1)l_i(x)} = x^3 + x^2 + 1
	\end{equation*}
	对上面两边求导,有
	\begin{align*}
		\sum_{i = 0}^6{(x_i^3 + x_i^2 + 1)l'_i(x)} & = \left(\sum_{i = 0}^6{(x_i^3 + x_i^2 + 1)l_i(x)}\right)' \\
		                                           & = (x^3 + x^2 + 1)'                                        \\
		                                           & = 3x^2 + 2x
	\end{align*}
	}

\end{question}

\end{document}
