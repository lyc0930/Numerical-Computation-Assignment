\documentclass[11pt]{article}
\usepackage[UTF8]{ctex}
\usepackage{subcaption}
%%%%%%%%%%%%%%%%%%%%%%%%%%%%%%%%%%%%%%%%%
% Cleese Assignment
% Structure Specification File
% Version 1.0 (27/5/2018)
%
% This template originates from:
% http://www.LaTeXTemplates.com
%
% Author:
% Vel (vel@LaTeXTemplates.com)
%
% License:
% CC BY-NC-SA 3.0 (http://creativecommons.org/licenses/by-nc-sa/3.0/)
%
%%%%%%%%%%%%%%%%%%%%%%%%%%%%%%%%%%%%%%%%%

%----------------------------------------------------------------------------------------
%	PACKAGES AND OTHER DOCUMENT CONFIGURATIONS
%----------------------------------------------------------------------------------------

\usepackage{cite}

\usepackage{lastpage} % Required to determine the last page number for the footer

\usepackage{graphicx} % Required to insert images

\setlength\parindent{0pt} % Removes all indentation from paragraphs

\usepackage[most]{tcolorbox} % Required for boxes that split across pages

\usepackage{booktabs} % Required for better horizontal rules in tables

\usepackage{listings} % Required for insertion of code

\usepackage{etoolbox} % Required for if statements

\usepackage{amsmath}
% 增加矩阵两侧括号内侧的空白,新建附参数arraystretch的matrix命令族
\makeatletter
\renewenvironment{pmatrix}
{\left(\mkern10.0mu\env@matrix}
{\endmatrix\mkern10.0mu\right)}
\renewenvironment{vmatrix}
{\left|\mkern10.0mu\env@matrix}
{\endmatrix\mkern10.0mu\right|}
\renewcommand*\env@matrix[1][\arraystretch]{%
    \edef\arraystretch{#1}%
    \hskip -\arraycolsep
    \let\@ifnextchar\new@ifnextchar
    \array{*\c@MaxMatrixCols c}}
\makeatother

\usepackage{amsthm}
\usepackage{amssymb}
\usepackage{indentfirst}
\usepackage{diagbox}
\usepackage{float}
\usepackage{xcolor}
\usepackage[colorlinks, linkcolor = black]{hyperref}

\usepackage{enumerate}
\usepackage{enumitem}
\setlist{
    leftmargin = .1\linewidth,
    % rightmargin = .1\linewidth,
    % label=\emph{\alph*}.
}

\setlength{\parindent}{2em}

\usepackage{siunitx}
\sisetup
{
    output-exponent-marker = \ensuremath{\mathrm{E}},
    exponent-product = {},
    retain-explicit-plus,
    retain-zero-exponent,
}
%----------------------------------------------------------------------------------------
%	MARGINS
%----------------------------------------------------------------------------------------

\usepackage{geometry} % Required for adjusting page dimensions and margins

\geometry{
    paper=a4paper, % Change to letterpaper for US letter
    top=3cm, % Top margin
    bottom=3cm, % Bottom margin
    left=2.5cm, % Left margin
    right=2.5cm, % Right margin
    headheight=14pt, % Header height
    footskip=1.4cm, % Space from the bottom margin to the baseline of the footer
    headsep=1.2cm, % Space from the top margin to the baseline of the header
    %showframe, % Uncomment to show how the type block is set on the page
}

%----------------------------------------------------------------------------------------
%	FONT
%----------------------------------------------------------------------------------------

\usepackage[utf8]{inputenc} % Required for inputting international characters
\usepackage[T1]{fontenc} % Output font encoding for international characters

% \usepackage[sfdefault,light]{roboto} % Use the Roboto font

%----------------------------------------------------------------------------------------
%	HEADERS AND FOOTERS
%----------------------------------------------------------------------------------------

\usepackage{fancyhdr} % Required for customising headers and footers

\pagestyle{fancy} % Enable custom headers and footers

\lhead{\small\assignmentClass} % Left header; output the instructor in brackets if one was set
\chead{\small\assignmentTitle} % Centre header
\rhead{\small\ifdef{\assignmentAuthorName}{\assignmentAuthorName}{\ifdef{\assignmentDate}{Due\ \assignmentDate}{}}} % Right header; output the author name if one was set, otherwise the due date if that was set

\lfoot{} % Left footer
\cfoot{\small Page\ \thepage\ of\ \pageref{LastPage}} % Centre footer
\rfoot{} % Right footer

\renewcommand\headrulewidth{0.5pt} % Thickness of the header rule

%----------------------------------------------------------------------------------------
%	MODIFY SECTION STYLES
%----------------------------------------------------------------------------------------

\usepackage{titlesec} % Required for modifying sections

%------------------------------------------------
% Section

\titleformat
{\section} % Section type being modified
[block] % Shape type, can be: hang, block, display, runin, leftmargin, rightmargin, drop, wrap, frame
{\Large\bfseries} % Format of the whole section
{\arabic{section}} % Format of the section label
{6pt} % Space between the title and label
{} % Code before the label

\titlespacing{\section}{0pt}{0.5\baselineskip}{0.5\baselineskip} % Spacing around section titles, the order is: left, before and after

%------------------------------------------------
% Subsection

\titleformat
{\subsection} % Section type being modified
[hang] % Shape type, can be: hang, block, display, runin, leftmargin, rightmargin, drop, wrap, frame
{\large\bfseries} % Format of the whole section
{\arabic{section}.\arabic{subsection}} % Format of the section label
{4pt} % Space between the title and label
{} % Code before the label

\titlespacing{\subsection}{0pt}{0.5\baselineskip}{0.5\baselineskip} % Spacing around section titles, the order is: left, before and after

\renewcommand\thesubsection{\arabic{section}.\arabic{subsection}}

%----------------------------------------------------------------------------------------
%	CUSTOM QUESTION COMMANDS/ENVIRONMENTS
%----------------------------------------------------------------------------------------



% Command to print an assignment section title to split an assignment into major parts
\newcommand{\assignmentSection}[1]{
    \newpage
    {
        \centering % Centre the section title
        \vspace{2\baselineskip} % Whitespace before the entire section title

        \rule{0.8\textwidth}{0.5pt} % Horizontal rule

        \vspace{0.75\baselineskip} % Whitespace before the section title
        {\LARGE \textsc{#1}} % Section title, forced to be uppercase

        \rule{0.8\textwidth}{0.5pt} % Horizontal rule

        \vspace{\baselineskip} % Whitespace after the entire section title
    }
    \setcounter{section}{0}

}

%----------------------------------------------------------------------------------------
%	TITLE PAGE
%----------------------------------------------------------------------------------------

\author{\textbf{\assignmentNo\ \assignmentAuthorName}} % Set the default title page author field
\date{} % Don't use the default title page date field

\title{
    \thispagestyle{empty} % Suppress headers and footers
    \vspace{0.2\textheight} % Whitespace before the title
    \textbf{\assignmentClass}\\[5pt]
    \texttt{\assignmentTitle}\\[-4pt]
    % \ifdef{\assignmentSubTitle}{\texttt{\assignmentSubTitle}}{}
    \ifdef{\assignmentDate}{\assignmentDate}{} % If a due date is supplied, output it
    \ifdef{\assignmentClassInstructor}{{\large \textit{\assignmentClassInstructor}}}{} % If an instructor is supplied, output it
    \vspace{0.32\textheight} % Whitespace before the author name
}


\sisetup
{
    table-format = +1.12e+003,
}

\newcommand{\assignmentQuestionName}{Question} % The word to be used as a prefix to question numbers; example alternatives: Problem, Exercise
\newcommand{\assignmentClass}{计算方法B} % Course/class
\newcommand{\assignmentTitle}{Programming Assignment\ \#5}
\newcommand{\assignmentDate}{2020.5.19} % date
\newcommand{\assignmentNo}{PB17000297}
\newcommand{\assignmentAuthorName}{罗晏宸} % Student name

\begin{document}

\maketitle
\thispagestyle{empty}
\newpage

\assignmentSection{复化积分}
\section{问题描述}
分别编写用复化 Simpson 积分公式和复化梯形积分公式计算积分的通用程序。

用如上程序计算积分 $I(f) = \displaystyle \int_0^8\!\sin{(x)}\,\text{d}x$。取等距节点,记节点$\{x_i,\ i = 0,\,\cdots,\,N\}$,其中$N$为$\left\{2^k,\ k = 0,\,1,\,\cdots,\,10\right\}$,并计算误差(用科学计数形式),同时给出误差阶(用浮点形式,比如$1.8789$)。比较并分析两种方法的优劣。

\paragraph{误差阶} 记步长为$h$时的误差为$\widetilde{e}$,步长为$h/n$时的误差为$\widetilde{e}_n$(这里$n = 2$),则相应的误差阶为:
\begin{equation*}
    d = - \frac{\ln{\left(\dfrac{\widetilde{e}_n}{\widetilde{e}}\right)}}{\ln{(n)}}
\end{equation*}

\section{计算结果}
由 C++ 计算得到结果按格式输出并列表如下,
\begin{figure}[h]
    % \centering
    % \begin{subfigure}{\textwidth}
    \centering
    \begin{tabular}{|l|c|c|}
        \hline
                           & 误差                                              & 误差阶                        \\ \hline
        $k = \phantom{1}0$ & $e_{0\phantom{1}} = 2.811932952685\text{E}{+}000$ &                               \\\hline
        $k = \phantom{1}1$ & $e_{1\phantom{1}} = 2.193993521794\text{E}{+}000$ & $d_{1\phantom{1}} = 0.358003$ \\   \hline
        $k = \phantom{1}2$ & $e_{2\phantom{1}} = 4.099829205476\text{E}{-}001$ & $d_{2\phantom{1}} = 2.419924$ \\   \hline
        $k = \phantom{1}3$ & $e_{3\phantom{1}} = 9.708816025468\text{E}{-}002$ & $d_{3\phantom{1}} = 2.078197$ \\   \hline
        $k = \phantom{1}4$ & $e_{4\phantom{1}} = 2.396461540682\text{E}{-}002$ & $d_{4\phantom{1}} = 2.018390$ \\   \hline
        $k = \phantom{1}5$ & $e_{5\phantom{1}} = 5.972370007437\text{E}{-}003$ & $d_{5\phantom{1}} = 2.004530$ \\   \hline
        $k = \phantom{1}6$ & $e_{6\phantom{1}} = 1.491925067877\text{E}{-}003$ & $d_{6\phantom{1}} = 2.001128$ \\   \hline
        $k = \phantom{1}7$ & $e_{7\phantom{1}} = 3.729084041566\text{E}{-}004$ & $d_{7\phantom{1}} = 2.000282$ \\   \hline
        $k = \phantom{1}8$ & $e_{8\phantom{1}} = 9.322254870159\text{E}{-}005$ & $d_{8\phantom{1}} = 2.000070$ \\   \hline
        $k = \phantom{1}9$ & $e_{9\phantom{1}} = 2.330535267947\text{E}{-}005$ & $d_{9\phantom{1}} = 2.000018$ \\   \hline
        $k = 10$           & $e_{10} = 5.826320388591\text{E}{-}006$           & $d_{10} = 2.000004$           \\ \hline
    \end{tabular}
    \caption{复化梯形积分误差及误差阶}
    \label{table:Trapezoid}
\end{figure}
% \end{subfigure}
% \\[5em]
% \begin{subfigure}{\textwidth}
\begin{figure}[h]
    \centering
    \begin{tabular}{|l|c|c|}
        \hline
                            & 误差                                              & 误差阶                                   \\ \hline
        $k = \phantom{1} 0$ & $e_{0\phantom{1}} = 1.492788623854\text{E}{+}000$ &                                          \\ \hline
        $k = \phantom{1} 1$ & $e_{1\phantom{1}} = 3.862635679953\text{E}{+}000$ & $d_{1\phantom{1}} = -1.371576$           \\ \hline
        $k = \phantom{1} 2$ & $e_{2\phantom{1}} = 1.846872798678\text{E}{-}001$ & $d_{2\phantom{1}} = \phantom{-}4.386429$ \\ \hline
        $k = \phantom{1} 3$ & $e_{3\phantom{1}} = 7.210093176285\text{E}{-}003$ & $d_{3\phantom{1}} = \phantom{-}4.678923$ \\ \hline
        $k = \phantom{1} 4$ & $e_{4\phantom{1}} = 4.098995424706\text{E}{-}004$ & $d_{4\phantom{1}} = \phantom{-}4.136676$ \\ \hline
        $k = \phantom{1} 5$ & $e_{5\phantom{1}} = 2.504512568935\text{E}{-}005$ & $d_{5\phantom{1}} = \phantom{-}4.032669$ \\ \hline
        $k = \phantom{1} 6$ & $e_{6\phantom{1}} = 1.556578642870\text{E}{-}006$ & $d_{6\phantom{1}} = \phantom{-}4.008079$ \\ \hline
        $k = \phantom{1} 7$ & $e_{7\phantom{1}} = 9.715041615621\text{E}{-}008$ & $d_{7\phantom{1}} = \phantom{-}4.002014$ \\ \hline
        $k = \phantom{1} 8$ & $e_{8\phantom{1}} = 6.069783342610\text{E}{-}009$ & $d_{8\phantom{1}} = \phantom{-}4.000503$ \\ \hline
        $k = \phantom{1} 9$ & $e_{9\phantom{1}} = 3.793281244668\text{E}{-}010$ & $d_{9\phantom{1}} = \phantom{-}4.000127$ \\ \hline
        $k = 10$            & $e_{10} = 2.370814655706\text{E}{-}011$           & $d_{10} = \phantom{-}3.999992$           \\ \hline
    \end{tabular}
    \caption{复化 Simpson 积分误差及误差阶}
    \label{table:Simpson}
    % \end{subfigure}
\end{figure}

\section{结果分析}
从误差结果来看,随着积分区间节点数的增加,两种复化积分的误差都逐渐减小,但复化 Simpson 积分的误差随着节点数的增加明显小于复化梯形积分的误差;从误差阶结果来看,两种复化积分的误差阶分别趋近于常数 2 和 4,由定义
\begin{equation*}
    d = - \frac{\ln{\left(\dfrac{\widetilde{e}_n}{\widetilde{e}}\right)}}{\ln{(n)}} = \log_n{\left(\dfrac{\widetilde{e}}{\widetilde{e}_n}\right)}
\end{equation*}
误差阶实际上指出了数值积分的误差下降速度对区间数$n$的对数,计算结果表明复化梯形积分的误差下降速度大致为 $n^{-2}$,而复化 Simpson 积分的误差下降速度大致为 $n^{-4}$。

\section{算法分析}
对于复化梯形数值积分,截断误差由
$$
    E_n(f) = I(f) - T_n(f) = -\frac{h^3}{12}\sum_{i = 0}^{n - 1}{f''(\xi_i)}
$$
给出,因为$\sin{x} \in C^2[0, 8]$,所以存在$\xi \in [0, 8]$,有
\begin{align*}
    E_n(\sin) & = -\frac{h^3}{12}\sum_{i = 0}^{n - 1}{\sin''(\xi_i)} \\
              & = \frac{h^3}{12} n \sin(\xi)                         \\
              & = \frac{h^2}{12} (8 - 0) \sin(\xi)                   \\
              & = \frac{2h^2}{3}\sin(\xi)                            \\
              & = \frac{128}{3n^2}\sin(\xi)                          \\
\end{align*}
由此可见复化梯形积分误差的截断误差按$h^2$(或$\dfrac{1}{n^2}$)的下降速度下降。

对于复化 Simpson 数值积分,设$n = 2m$,所以存在$\zeta \in [0, 8]$,有
\begin{align*}
    E_n(f) & = I(f) - S_n(f)                                                                                                                                     \\
           & = \sum_{i = 0}^{m - 1}\int_{x_{2i}}^{x_{2i+2}}\!f(x)\,\text{d}x - S_n(f)                                                                            \\
           & = \sum_{i = 0}^{m - 1}\left\{\frac{2h}{6}\left[f(x_{2i}) + 4f(x_{2i+1}) + f(x_{2i+2})\right] - \frac{(2h)^5}{2880}f^{(4)}(\zeta_i)\right\} - S_n(f) \\
           & = \sum_{i = 0}^{m - 1}\left\{- \frac{(2h)^5}{2880}f^{(4)}(\zeta_i)\right\}                                                                          \\
           & = - \frac{(2h)^5}{2880}\sum_{i = 0}^{m - 1}f^{(4)}(\zeta_i)                                                                                         \\
           & = - \frac{(2h)^5}{2880} m \sin(\zeta)                                                                                                               \\
           & = - \frac{2h^4}{45}\sin(\zeta)                                                                                                                      \\
           & = - \frac{8192}{45n^4}\sin(\zeta)
\end{align*}
由此可见复化 Simpson 积分误差的截断误差按$h^4$(或$\dfrac{1}{n^4}$)的下降速度下降。

结合计算结果,两组数据的误差阶都与理论值符合地较好。

\section{实验结论}
实验通过对同一区间上正弦函数的数值积分计算,比较了复化梯形积分和复化 Simpson 积分两种方法的计算结果、数值误差及其误差阶。通常情况下,数值积分公式的代数精度越高,计算精度也越高,复化 Simpson 积分的误差较复化梯形积分小,近似效果较好。

\end{document}